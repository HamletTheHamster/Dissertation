\chapter{Conclusion}
\label{ch:Conclusion}
\acresetall

Throughout this dissertation, we have explored how Brillouin-based optomechanics can be harnessed for both practical device applications and fundamental studies of light–matter interactions. Specifically, we demonstrated that traveling-wave phonons in an extended medium, such as a liquid-core optical fiber, can be cooled significantly through the anti-Stokes Brillouin scattering process. This result marks an important step in transitioning from the established model of discrete cavity modes toward the more general case of continuous-waveguide phonons, showing that key regimes of laser cooling can be accessed without relying on highly engineered photonic microstructures.

In tandem, this dissertation introduced the concept of a coherently stimulated Brillouin spectrometer (CoBS), which relaxes traditional phase-matching constraints and dramatically increases measurement sensitivity for short-length or low-gain samples. By incorporating a four-wave mixing scheme—pump, probe, driven Stokes, and backscattered signal—the spectrometer opens new avenues for measuring the mechanical modes of thin films, microscale liquid volumes, and emerging photonic–phononic waveguides. This flexibility is crucial for advancing applications ranging from precision material characterization to integrated acousto-optic devices.

Beyond traveling-wave phonons, we also proposed how Brillouin-induced Raman-like modes might be generated and observed, potentially bridging the gap between conventional Raman spectroscopy and Brillouin acoustics. Although direct experimental realizations of these hybrid modes remain at an early stage, the theoretical groundwork and device prototypes described here highlight the promise of such approaches for both fundamental and applied research. Ultimately, these efforts underscore the rich variety of phononic phenomena accessible when one leverages strong acousto-optic interactions in engineered media.

Looking ahead, an important direction will be to pursue net phonon cooling in waveguides by minimizing or routing away the unwanted Stokes contributions. Achieving such deep cooling could open pathways to ground-state phonon occupancy, enabling quantum acoustic memory and photon–phonon entanglement protocols. In tandem, refining the CoBS method to push detection limits and spatial resolution even further may help unlock additional device architectures across the fields of quantum information, ultralow-noise photonics, and advanced sensing. In this way, the work presented herein lays the foundation for continued innovation at the intersection of photonics, phononics, and materials science.

\clearpage
\thispagestyle{empty}
\null
\newpage
