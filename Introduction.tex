\chapter{Introduction}
\label{ch:Introduction}
\acresetall

%TOO SELF INTERESTED! State facts.
%The work presented in this document details my efforts toward achieving a result for which my time and available resources would prove too limited to accomplish in full. It is my great hope that a dear future reader may someday be inspired to take up where I left off and realize the result that I have been chasing for these years. No physical principle disallows the demonstration of this result, only practical challenges stand in its way.

%The physics involved in this research effort primarily lies in the field of optomechanics, however it veers at times into the related domains of nonlinear optics and nanoscience. I have in fact developed a great love for each of these fields and it was my romantic goal to unite them that originally led me to pursue this work. In this introductory chapter I present the foundational topics and concepts that I employ in later chapters.

Optomechanics is the study of light-matter interactions; it is the study of how the intangible (light) can affect change in the tangible (matter) and visa-versa. Injecting light into a material under specific conditions allows for an exchange of energy to occur between the light and the mechanical oscillations of the material which changes the mechanical energy of the material. This interaction can be controlled to deposit or withdraw mechanical energy into/from a system and thus leave the system in a more, or less, mechanically energizetic state respectively. The same interaction can be harnessed instead for passive observation of material properties. Mechanical systems from bulk to atomic scales can be probed and characterized with light by retrieving the inelastically scattered light resulting from interaction with the material. This retrieved light contains embedded information about the energy exchange that occured which, when considered as part of a population of scattering events, reveals natural resonances of a mechanical system.


%%%%%%%%%%%%%%%%%%%%%%%%%%%%%%%%%%%
\section{Spontaneous Brillouin Scattering}
\label{sec:Introduction:Spontaneous}
\lipsum[1]

\section{Stimulated Brillouin Scattering}
\label{sec:Introduction:Stimulated}
\lipsum[1]

\section{Phase-matching}
\label{sec:Introduction:Phase-matching}
\lipsum[1]

\section{Brillouin Gain of Materials}
\label{subsec:Introduction:Gain}

\section{Raman Scattering}
\label{sec:Introduction:Raman}
\lipsum[1]

\section{Raman-like Brillouin Modes}
\label{sec:Introduction:Raman-like}
\lipsum[1]
