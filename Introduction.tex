\chapter{Introduction}
\label{ch:Introduction}
\acresetall

This is an inline citation, \cite{boyd2020nonlinear}. This is a parenthetical citation \citep{boyd2020nonlinear}. This is a figure reference (Figure \ref{fig:cooling-system}). This is a section reference \S\ref{sec:Ch1:Brillouin Scattering}. This is a chapter reference with chapter spelled out: \autoref{chap: 2-Cooling}. This is an acronym definition \ac{AGU}. This is the second time I use the acronym in this section \ac{AGU}. This is if I want to spell out the full acronym again \acf{AGU}. Define new acronyms in the acronyms.tex file.


%%%%%%%%%%%%%%%%%%%%%%%%%%%%%%%%%%%
\section{Spontaneous Brillouin Scattering}
\label{sec:Introduction:Spontaneous}
\lipsum[1]

\section{Stimulated Brillouin Scattering}
\label{sec:Introduction:Stimulated}
\lipsum[1]

\section{Phase-matching}
\label{sec:Introduction:Phase-matching}
\lipsum[1]

\section{Brillouin Gain of Materials}
\label{subsec:Introduction:Gain}

\section{Raman Scattering}
\label{sec:Introduction:Raman}
\lipsum[1]
