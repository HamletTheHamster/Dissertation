\setcounter{rownumber}{0}
\singlespacing
\chapter{Brillouin-Induced Raman Modes and Device Exploration}
\label{ch:Raman}
\acresetall

\doublespacing

%%%%%%%%%%%%%%%%%%%%%%%%%%%%%%%%%%%%%%%%%%%%%%%%%%%%%%%%%%%%%%%%%%%%

\section{Introduction}
\label{sec:Raman:Introduction}

This chapter explores progress towards the goal of demonstrating Brillouin-induced Raman-like modes at room temperature. We aim to show that acoustic traveling-wave phonons generated by a Brillouin scattering process in a confined medium can organize into discrete standing-wave vibrational modes. This goal represents a convergence of several themes in modern photonics: cavity optomechanics, coherent phonon control, and nonlinear light–matter interactions. Cavity optomechanics has traditionally focused on discrete mechanical resonances such as drumhead membranes or whispering-gallery (\ac{WGM}) microresonators to achieve effects like laser cooling of vibrational modes or phonon lasing. \cite{kippenberg2008cavity, chan2011laser, aspelmeyer2014cavity, vahala2009phonon} In parallel, Brillouin scattering provides a route to manipulate traveling-wave acoustic phonons in extended media, as demonstrated by our Brillouin-based laser cooling in optical fiber (a “cavity-less” system). \cite{johnson2023laser, eggleton2013inducing, bahl2012observation, otterstrom2018optomechanical} The research in this chapter aims to bridge these domains, leveraging traveling-wave Brillouin scattering in a finite-length system so that the phonons become self-confined, resembling the vibrational modes that give rise to Raman scattering.

This investigation is grounded in the broader context of controlling phonons and light at the mesoscale. High-coherence phonons are garnering interest for precision metrology and quantum information, \cite{balram2016coherent, schliesser2014cavity} spurring new strategies for phonon coherent optomechanical interactions. \cite{kippenberg2008cavity, aspelmeyer2014cavity} Earlier chapters of this dissertation advanced this frontier: Chapter~\ref{ch:Cooling} demonstrated the laser cooling of propagating acoustic phonons in an optical fiber, extending optomechanical control to continuous media at room temperature, \cite{johnson2023laser} and Chapter~\ref{ch:CoBS} introduced a novel Brillouin spectrometer which offers especially high sensitivity for short (\(<\)\SI{1}{\centi\meter}) lengths over traditional \ac{SBS}. Building on such results, the present work tackles the next challenge: inducing standing-wave acoustic modes in a bulk-like sample through Brillouin processes. Achieving this at room temperature would be significant, as to date strong acoustic mode formation has been mostly limited to cryogenic systems where phonon lifetimes are long. \cite{otterstrom2018optomechanical, galliou2013extremely} By pursuing Brillouin-induced modes under ambient conditions, we push toward practical phonon devices and new regimes of light–sound interaction without the strict need for optical cavities. \cite{pant2011chip}

%--------------------------------------------------------------------%

\section{From Traveling-Wave to Raman-Like Standing-Wave Modes}
\label{sec:Raman:FromTraveling-WavetoRaman-LikeStanding-WaveModes}

\subsection{Review of Brillouin and Raman Scattering}
\label{subsec:Raman:ReviewofBrillouinandRamanScattering}

Brillouin scattering and Raman scattering are two related light–matter interactions that involve inelastic scattering of photons by phonons, but they differ in the nature of the phonons involved. In Brillouin scattering, an incident photon exchanges energy and momentum with a long-wavelength acoustic phonon (a propagating sound wave in the medium). In contrast, Raman scattering typically involves optical phonons or molecular vibrations, which are localized oscillations (e.g., bond vibrations within a molecule or internal lattice vibrations) rather than a continuum acoustic wave. In other words, Brillouin scattering is mediated by traveling acoustic waves in a bulk material, whereas Raman scattering probes standing-wave intramolecular or lattice modes. This fundamental difference is reflected in the frequency scales: acoustic phonons have relatively low frequencies (\si{\giga\hertz} or below) and are responsible for the small Stokes/anti-Stokes shifts in Brillouin spectra, whereas optical phonons and molecular vibrations have much higher frequencies (\si{\tera\hertz}) yielding the larger shifts seen in Raman spectra (see Figure~\ref{fig:Introduction:scattering-domains}). \cite{cardona2007light} It is also reflected in momentum conservation conditions: Brillouin scattering requires phase-matching between the optical wave and an acoustic phonon of a particular wavevector, essentially picking out a traveling phonon mode with a definite momentum. Raman scattering, on the other hand, often involves phonons with near-zero wavevector (e.g., zone-center optical modes in crystals or whole-molecule vibrations) due to the selection rule of crystal momentum conservation in perfect lattices. \cite{ferraro2003introductory}

Despite these distinctions, the line between Brillouin and Raman processes can blur in certain situations. If the medium lacks long-range order or the phonon coherence length is short, the usual momentum-selection rules break down. Shuker and Gammon (1970) \cite{shuker1970raman} showed that in amorphous materials, translational symmetry is lost and essentially all vibrational modes can participate in light scattering. In such cases, even low-frequency acoustic-like modes (normally the realm of Brillouin) become “Raman-active.” This insight explained the observation of broad low-frequency Raman scattering (the so-called boson peak) in glasses by attributing it to acoustic vibrations made allowable by disorder. \cite{duval1990vibrational, winterling1975very, nemanich1977low, martin1974model, malinovsky1986nature, buchenau1986low, malinovsky1987investigation, chumakov2011equivalence} Conversely, in a small or confined system, the vibrational modes are discrete rather than forming a continuous acoustic band. In effect, they approach the molecular limit where each normal mode can scatter light akin to a Raman transition. Thus, one can view Brillouin and Raman scattering as two limits of the same fundamental interaction, distinguished by whether the phonons act as continuous waves or as localized normal modes.

Spatial confinement of acoustic waves can convert the Brillouin regime into a Raman-like regime. In an unbounded or large medium, acoustic phonons exist over a continuum of frequencies and wavevectors (a traveling-wave picture). But if the medium is bounded (e.g., a micron-scale particle or a cavity of finite length) the acoustic field must satisfy boundary conditions, leading to a set of allowed eigenmodes. Classic elasticity theory by Lamb (1882) \cite{lamb1881vibrations} provided the first analysis of this: a homogeneous elastic sphere supports only certain quantized vibration modes (classified into spheroidal and torsional families) dictated by the sphere’s finite size and free-surface boundary condition. These quantized vibration modes are true standing-wave modes confined to the sphere. Subsequent work extended Lamb’s theory to include various effects such as surface tension and clamping for small particles, \cite{} but the essential result is that a finite object has discrete acoustic eigenmodes. Experimentally, Duval et al. (1986) \cite{duval1986vibration} performed a notable demonstration by observing very-low-frequency Raman scattering from nanometer-sized microcrystals embedded in glass. The Raman spectra showed distinct peaks whose frequencies scaled inversely with the particle diameter, exactly as expected for Lamb’s vibrational modes in a sphere. Duval identified these peaks as the confined acoustic eigenmodes (“particle vibrational modes”) of the microcrystallites, which had become Raman-active. This exemplified how a Brillouin-like acoustic wave (here, a sphere’s breathing or shear wave) can become Raman-like when spatially confined. When an acoustic phonon is restricted by boundaries, it is no longer a freely propagating wave but rather a normal mode with a discrete frequency, effectively turning a Brillouin interaction into a Raman-style interaction with a set of allowed modes.

Figure~\ref{fig:Raman:BrillouinRamanTransition} offers a conceptual illustration of the transition from a traveling acoustic wave to standing-wave vibrational modes. In a bulk material (left), light scatters from a continuum of acoustic waves (Brillouin scattering), whereas in a confined medium (right), only discrete phonon modes are allowed, producing Raman-like spectral lines. Spatial confinement and acoustic reflections thus shift the scattering from the Brillouin regime toward a Raman-like regime. Contemporary research in cavity optomechanics and related fields has leveraged this wave-to-mode transition as well. For instance, Renninger et al. (2018) \cite{renninger2018bulk} showed that by shaping the geometry of a \si{\centi\meter}-scale crystal, one can support long-lived acoustic standing-waves even in a bulk solid. At low temperature, the phonon coherence length in their system exceeded the system size, and the crystal effectively behaved as a giant “phonon cavity” with high-Q (quality) acoustic modes. \cite{renninger2018bulk, maccabe2020nano} These modes were accessed via Brillouin interactions, blurring the line between traditional Brillouin and Raman: the process was Brillouin in origin (photoelastic coupling to sound waves) but the phonons were in discrete cavity modes like Raman vibrations. These studies underscore that confined acoustic phonons can take on the character of Raman modes.

\begin{figure}[t]
  \centering
  \includegraphics[width=\textwidth]{figs/4-Raman/ExploreBrillouinRamanTransition.png}
  \vspace{0.5em}
  \caption{Conceptual illustration of the transition from a traveling acoustic wave to standing-wave vibrational modes. In a bulk material (left), light scatters from a continuum of (traveling) acoustic waves (Brillouin scattering), whereas in the molecular vibrations of atomic bonds (right), only discrete (standing-wave) phonon modes are allowed. The left diagram is a conceptual visualization of the \ac{CoBS} process for a longitudinally-traveling phonon \(\Omega\) while the right diagram gives a conceptual visualization of an analogous \ac{CARS} process with the standing-vibrations \(\omega_{\rm v}\) among bonded atoms in a molecule.}
  \label{fig:Raman:BrillouinRamanTransition}
\end{figure}

\subsection{Brillouin-Induced Raman Modes}
\label{subsec:Raman:Brillouin-InducedRamanModes}

In a traditional SBS experiment, a pump laser drives an acoustic wave through electrostriction, and the scattered Stokes light is down-shifted by the acoustic frequency \(f_{\rm B}\), given by \cite{boyd2020nonlinear}

\begin{equation}
  f_{\rm B} = \frac{2v_{\rm s}n}{\lambda},
  \label{eq:Raman:f_B}
\end{equation}

where \(v_{\rm s}\) is the longitudinal sound speed, \(n\) is the refractive index, and \(\lambda\) is the optical frequency. In an unbounded or long medium, the frequency response of the scattered light is determined by material properties (sound speed and optical dispersion) and the acoustic wave can be thought of as a traveling grating moving through the medium. If the medium is shortened such that the acoustic wave can reflect off the sample boundaries, the traveling acoustic wave can auto-interfere with itself to form a standing-wave pattern in the medium. Under these system conditions, a phonon induced by \ac{SBS} will propagate to the sample end, reflect (assuming a high acoustic impedance mismatch at the boundary), and traverse the medium in the reverse direction. If the length \(L\) between acoustic interfaces is a half-integer number of acoustic wavelengths, the forward and backward phonon waves can interfere to form a standing wave pattern (i.e., a resonance), in essence trapping the traveling phonon in the cavity defined by the sample. This standing wave in turn acts like a stable grating, enhancing the scattering of light at well-defined frequencies corresponding to its resonances. We term these resonant phonon excitations “Brillouin-induced Raman modes” to reflect their hybrid character.

In more concrete terms, the finite-length system behaves like an acoustic Fabry–Pérot cavity. For a longitudinal acoustic mode, the condition for a standing wave is that an integer number \(n\) of half-wavelengths equals the round-trip length: \(n \cdot (\lambda_{\rm s}/2) = L\). Equivalently, the allowed acoustic frequencies are

\begin{equation}
  f_{\rm n} = \frac{n\,v_{\rm s}}{2L},
  \label{eq:Raman:f_R}
\end{equation}
\\
where \(v_{\rm s}\) is the sound velocity in the medium and \(n = 1, 2, 3, \dots\) indexes the mode order. The lowest-frequency mode (\(n = 1\)) has a fundamental frequency \(f_{\rm 1} \approx v_{\rm s}/(2L)\), and higher modes are integer multiples of this fundamental (assuming a simple one-dimensional confinement). Thus, instead of a single Brillouin shift \(f_{\rm B}\), one expects a ladder of equally spaced phonon modes in the scattering spectrum, resembling a Raman vibrational progression. The spacing \(\Delta f\) between adjacent modes is approximately \(\Delta f \approx v_{\rm s}/(2L)\), set solely by the cavity length and sound speed. Figure~\ref{fig:Raman:GeometryDeterminesFundamentalFreq} illustrates why this is so. For example, if \(L =\) \SI{10}{\milli\meter} in a medium where \(v_{\rm s} =\) \SI{5000}{\meter\per\second}, the fundamental mode would be \(f_{\rm 1} \sim\) \SI{250}{\kilo\hertz} and overtones at \SI{500}{\kilo\hertz}, \SI{750}{\kilo\hertz}, etc. In principle, a sufficiently short and high-Q acoustic cavity could produce a comb of multiple \si{\giga\hertz}-range lines (analogous to molecular vibrational Raman lines) out of what would ordinarily be a single broad Brillouin gain peak.

\begin{figure}[t]
  \centering
  \includegraphics[width=.85\textwidth]{figs/4-Raman/GeometryDeterminesFundamentalFreq.png}
  \caption{Illustration showing how geometry and sound speed of the material determine the allowed acoustic frequencies, given by Equation~\ref{eq:Raman:f_R}. Traveling-wave phonons around the material's Brillouin frequency shift \(f_{\rm B}\) are transduced within the medium via the \ac{CoBS} process. These phonons traverse the length to reach the boundary, where a high acoustic impedance mismatch causes them to reflect at the interface and retraverse the length of the medium in the reverse direction. The spatio-temporal overlap of counterpropagating traveling-wave phonons causes them to interfere, creating an acoustic standing-wave pattern within the material at discrete frequencies determined by the length and sound speed of the material. Shown in the illustration is the half-wavelength fundamental (\(n=1\)) mode, however in general, harmonics near the material's mechanical resonance frequency \(f_{\rm B}\) would be excited in the medium.}
  \label{fig:Raman:GeometryDeterminesFundamentalFreq}
\end{figure}

These Brillouin-induced modes would manifest as distinct peaks in the spectrum of the scattered light. In addition to the usual broadened Brillouin spectrum (with width given by acoustic damping), one would see a series of sharp lines at \(f_{\rm 1}, f_{\rm 2}, f_{\rm 3}, \dots\) around the center scattered frequency. Such a spectrum would be clear evidence that the phonon field is not only a freely propagating wave but also oscillating in discrete standing patterns (i.e., an optically driven acoustic resonator within the material). This is conceptually similar to stimulated Raman scattering in a molecule, where one can get a cascade of Stokes lines corresponding to \(1\hbar\omega_{\rm vib}\), \(2\hbar\omega_{\rm vib}\), \(3\hbar\omega_{\rm vib}\) energy shifts if the pump is intense. Here the “molecular vibration” is replaced by an acoustic cavity mode. If the pump power is high enough to drive the acoustic mode into the nonlinear regime, one might even observe multiple orders of Stokes (and anti-Stokes) as the phonon population builds up in those modes.

The idea of \ac{SBS}-driven acoustic modes has parallels in prior work. In the cryogenic experiment of Renninger et al. mentioned earlier, \cite{renninger2018bulk} a bulk acoustic mode was driven via Brillouin scattering in a \si{\centi\meter}-scale quartz crystal. At low temperatures, they observed ultra-narrow acoustic resonances (indicative of discrete modes) in place of a broad Brillouin response, confirming that acoustic coherence across the entire sample can indeed produce a modal spectrum. In our case, we seek to do this at room temperature by engineering a shorter effective acoustic cavity. We aim to accomplish this by taking advantage of our Coherently Stimulated Brillouin Spectrometer (\ac{CoBS}), which offers \(\sim10^{6}\) improvement in scattered power for \(\sim\)\si{\centi\meter} lengths over traditional \ac{SBS} (see Appendix~\ref{appendix:comparison} for a comparison of scattered power across Brillouin techniques, and specifically Figure~\ref{fig:SponBSvsStimBSvsCoBS} for a comparison by effective length \(L\)).

\subsection{Key Parameters and Feasibility}
\label{subsec:Raman:KeyParametersandFeasibility}

Observing Raman-like standing-wave modes via \ac{SBS} hinges on several key parameters of the system. We identify three especially critical factors: (1) the scattered power, determined by the effective material Brillouin gain and length as well as the optical powers; (2) the acoustic dissipation in the medium, which limits the mean phonon travel distance; and (3) the acoustic boundary reflectivity, determined by impedance mismatch at interfaces, which enables the phonons to reflect and form standing-waves. These parameters together determine whether the phonon will remain a distributed traveling excitation or collapse (blur) into discrete modes. Chapter~\ref{ch:CoBS} describes the \ac{CoBS} instrument, showing that the scattered power as a result of the \ac{CoBS} process is given by Equation~\ref{Eq:Theoretical Framework:Scattered Power}, stated here again as

\begin{equation}
  P_{\rm Signal} = \frac{1}{4}(G_{\rm B}L)^{2}P_{\rm Pump}P_{\rm Stokes}P_{\rm Probe}\Phi,
  \label{eq:Raman:ScatteredPowerPhi}
\end{equation}
\\
where \(G_{\rm B}\) is the material-dependent effective (acousto-optic overlap-adjusted) Brillouin gain factor given by Equation~\ref{Eq:Effective Brillouin Gain}, \(L\) is the effective length, \(P_{\rm i}\) are the optical powers of the pump, Stokes, and probe waves, respectively, and \(0 < \Phi < 1\) is a phase-matching relaxation term (given by Equation~\ref{Eq:Phi}) that captures the pump-probe detuning on which the instrument relies. A material’s Brillouin gain coefficient \(g_{\rm 0}\) (\si{\watt\per\meter}), or overlap‐adjusted effective gain \(G_{\rm B}\) (\si{\per\watt\per\meter}), sets how strongly the phonons are driven for given pump \(P_{\rm P}\), Stokes \(P_{\rm S}\), and probe \(P_{\rm Pr}\) optical powers over an interaction length \(L\) in the \ac{CoBS} process (see Chapter~\ref{ch:CoBS}, and specifically Equation~\ref{Eq:Theoretical Framework:Scattered Power}), given here again by

\begin{equation}
  G_{\rm B} = \frac{g_{0}}{A_{\rm eff}}\frac{\left(\frac{\Gamma_{\rm B}}{2}\right)^{2}}{\left(\Omega - \Omega_{\rm B}\right)^{2} + \left(\frac{\Gamma_{\rm B}}{2}\right)^{2}}.
  \label{eq:Raman:GB}
\end{equation}
\\
Here, \(\Gamma_{\rm B}\) is the angular Brillouin linewidth, \(\Omega\) (\(\Omega_{\rm B}\)) is the (resonant) angular acoustic frequency, \(A_{\rm eff}\) is the effective area (acousto-optic mode overlap), and \(g_{0}\) is the Brillouin gain coefficient given by

\begin{equation}
  g_{0} = \frac{\gamma_{\rm e}^{2}\omega^{2}}{nv_{\rm s}c^{3}\rho_{0}\Gamma_{\rm B}},
  \label{eq:Raman:g0}
\end{equation}
\\
where \(\gamma_{\rm e}\) is the electrostrictive constant, \(\omega\) is the angular optical frequency, \(n\) is the refractive index, \(v_{\rm s}\) is the speed of sound in the material, \(c\) is the speed of light, and \(\rho_{0}\) is the mean density of the material. Short samples demand very high effective Brillouin gain to achieve significant scattered power in a small \(L\). Certain tellurium‐based materials, for instance, can offer gains orders of magnitude higher than silica, \cite{sanghera2010nonlinear, abedin2005observation} allowing measureable scattered power in sub‐\si{\milli\meter} cavities.

Even if phonons are driven strongly, they must live long enough (i.e., have a low enough dissipation rate, or high enough acoustic Q) to form a standing wave. At room temperature, intrinsic damping can limit phonon \(Q_{\rm s}\) to \(\sim10^{3}\)–\(10^{4}\) in many solids, \cite{heiman1979brillouin, bucaro1974high} implying attenuation lengths of \si{\milli\meter} to \si{\centi\meter} for \si{\giga\hertz} frequencies. Ideally, the sample length \(L\) should be comparable to or less than half the attenuation length so that phonons undergo multiple round trips. This is considerably more difficult at room temperature than at cryogenic temperatures, where \(Q_{\rm s}\) can exceed \(10^{7}\). \cite{maris1990phonon, renninger2018bulk} Finally, the phonon must reflect rather than escape at the boundaries. A large acoustic impedance mismatch (e.g., a free surface with air on one side) can approach nearly 100\% reflection. \cite{galliou2013extremely, auld1973acoustic} Designing the sample with two opposing highly acoustically reflective boundaries is essential to generating a Raman-like standing-wave mode in the medium. In practice, however, partial reflections at each (or even just one) end can suffice so long as the net round‐trip reflectivity is high enough to sustain a mode.

In short, we want to create a high-\(Q\) acoustic resonator inside a Brillouin-active medium at room temperature: strong enough acousto-optic driving to excite the phonons, low enough damping to maintain them, and robust boundary reflections to confine them. Meeting all of these conditions can be challenging. However, the high (\(\sim\)\SI{5}{\femto\watt}) sensitivity and unique short path length advantage of our \ac{CoBS} instrument (described in Chapter~\ref{ch:CoBS} and Section~\ref{appendix:comparison}) sparks new motivation, as it provides a new technique tailored for observing small-scale scattering phenomenon. Feasibility estimates and initial \ac{CoBS} measurements indicate that by utilizing ultra-high-gain media such as \ce{Te} \cite{sanghera2010nonlinear, abedin2005observation} or liquid \ce{CS2} \cite{boyd2020nonlinear}, and ensuring at least one boundary is acoustically reflective, one can push toward Brillouin-induced Raman modes even under ambient conditions.

The sensitivity of our instrument, representing the minimum scattered power that may be detected, has been measured at \(P_{\rm Signal}\approx\) \SI{5}{\femto\watt} (see Section~\ref{Results:Instrument sensitivity}, and specifically Table~\ref{tab:CoBS:5fWSensitivity} together with Equation~\ref{Eq:Theoretical Framework:Scattered Power} and Figure~\ref{fig:CoBS:5fWSensitivity} for validation of this claim). Using this sensitivity value for \(P_{\rm Signal}\) we can rearrange Equation~\ref{eq:Raman:ScatteredPowerPhi} to solve for the minimum length \(L\) we can expect to observe scattering within for given optical powers, material gain, and pump-probe detuning:

\begin{equation}
  L = \frac{2}{G_{\rm B}}\sqrt{\frac{P_{\rm Signal}}{P_{\rm Pump}P_{\rm Stokes}P_{\rm Probe}\Phi}},
  \label{eq:Raman:minimumL}
\end{equation}
\\
where \(0 < \Phi < 1\) and increases for smaller \(L\). For \(P_{\rm Signal}\approx\)\SI{5}{\femto\watt} sensitivity and maximum optical powers \(P_{\rm Pump}P_{\rm Stokes}P_{\rm Probe}=\) \SI{0.25}{\cubic\watt} under typical conditions, Equation~\ref{eq:Raman:minimumL} predicts, for example, the ability to observe scattering within \(\sim\)\SI{500}{\nano\meter} of \ac{UHNA3} fiber (\(G_{\rm B,\,UHNA3}=\) \SI{0.6}{\per\watt\per\meter}). Equation~\ref{eq:Raman:minimumL} makes clear that minimizing the observable scattering length is accomplished by any of: bumping optical powers, improving instrument sensitivity, or choosing a higher gain scattering medium.

In what follows, we detail the experimental platforms and theoretical modeling that guided our attempts to observe discrete phonon modes in high-gain materials. Although a conclusive demonstration proved elusive, the conceptual framework is robust and provides a foundation for ongoing efforts. By shrinking the acoustic path length, maximizing phonon reflectivity, and exploiting strong \ac{CoBS} gain, one can approach the regime where traveling-wave Brillouin scattering morphs into Raman-like standing-wave modes. This pursuit effectively unifies the traveling-wave and standing-wave paradigms of light-sound interaction, providing new opportunities in cavity-free phononics, resonant optomechanics, and coherent phonon devices at room temperature.

\begin{figure}[t]
  \centering
  \includegraphics[width=\textwidth]{figs/4-Raman/GainofRelevantMaterials.png}
  \caption{Gain of relevant materials. \cite{dubinskii2004teo2}}
  \label{fig:Raman:GainofRelevantMaterials}
\end{figure}

%--------------------------------------------------------------------%

\section{Results}
\label{sec:Raman:Results}

Plots
\begin{itemize}
  \item UHNA3 - 1cm, 1mm
  \item CS2 vial - 4mm, 2mm
  \item TeO2 films - 1um, 500nm
  \item CS2 - 1mm, 100um, (10um not quite)
  \item chip waveguide - chip/nochip, holes
\end{itemize}

\subsection{Germanium-Doped Optical Fiber}
\label{subsec:Raman:Target:UHNA3}

\begin{itemize}
  \item 1cm, 1mm
\end{itemize}

\begin{figure}[t]
    \centering
    \begin{subfigure}[b]{0.49\textwidth}
        \centering
        \includegraphics[width=\textwidth]{figs/4-Raman/1cm UHNA3.jpeg}
        \caption{}
        \label{fig:Raman:1cmUHNA3pic}
    \end{subfigure}
    \hfill
    \begin{subfigure}[b]{0.49\textwidth}
        \centering
        \includegraphics[width=\textwidth]{figs/4-Raman/1mm UHNA3 in apparatus.jpeg}
        \caption{}
        \label{fig:Raman:1mmUHNA3pic}
    \end{subfigure}
    \caption{\SI{1}{\centi\meter} (\ref{fig:Raman:1cmUHNA3pic}) and \SI{1}{\milli\meter} (\ref{fig:Raman:1mmUHNA3pic}) \ac{UHNA3}.}
    \label{fig:Raman:UHNA3}
\end{figure}

\begin{figure}[t]
  \centering
  \hspace{-2em}\includegraphics[width=.85\textwidth]{figs/4-Raman/CoBS Measurement: 1 cm UHNA3.png}
  \caption{Observed background-subtracted spectrum obtained through a \ac{CoBS} measurement of \SI{1}{\centi\meter} of \ac{UHNA3} fiber (pictured in Figure~\ref{fig:Raman:1cmUHNA3pic}) captured at maximum operating optical powers \(P_{\rm P}P_{\rm S}P_{\rm Pr} \sim\) \SI{0.25}{\cubic\watt}, with 1\(\sigma\) uncertainties smaller than the data point markers.}
  \label{fig:Raman:1cmUHNA3}
\end{figure}

\begin{figure}[t]
  \centering
  \hspace{-2em}\includegraphics[width=.85\textwidth]{figs/4-Raman/CoBS Measurement: 1 mm UHNA3.png}
  \caption{Observed background-subtracted spectrum obtained through a \ac{CoBS} measurement of \SI{1}{\milli\meter} of \ac{UHNA3} fiber (pictured in Figure~\ref{fig:Raman:1mmUHNA3pic}) captured at maximum operating optical powers \(P_{\rm P}P_{\rm S}P_{\rm Pr} \sim\) \SI{0.25}{\cubic\watt}, with 1\(\sigma\) uncertainties smaller than the data point markers.}
  \label{fig:Raman:1mmUHNA3}
\end{figure}

\subsection{Free-Space Optics with Liquid Carbon Disulfide}
\label{subsec:Raman:Target:CS2Vial}

\begin{itemize}
  \item Free space with vial
\end{itemize}

\begin{table}[h]
    \centering
    \begin{tabular}{c c c c c c c c c}
        \toprule
        \textbf{\ce{CS2}} &
        \(\mathbf{\Gamma_{\rm \textbf{B}}}\) \cite{boyd2020nonlinear, johnson2023laser, enright1974depolarized, coakley1975brillouin} &
        \(\mathbf{\tau}\) &
        \(\mathbf{v_{\rm \textbf{s,\,long}}}\) \cite{boyd2020nonlinear, johnson2023laser, behunin2019spontaneous, geilen2023extreme} &
        \(\mathbf{n}\) \cite{boyd2020nonlinear, johnson2023laser} &
        \(\mathbf{L_{\rm \textbf{coh}}}\) &
        \(\mathbf{P_{\rm \textbf{CoBS,\,\(L_{\rm coh}/2\)}}}\) &
        \(\mathbf{\Omega_{\rm \textbf{B}}}\) &
        \(\mathbf{\Omega_{\rm \textbf{R,\,\SI{1}{\micro\meter}}}}\) \\
        &
        (\si{\mega\hertz}) &
        (\si{\nano\second}) &
        (\si{\meter\per\second}) &
        &
        (\si{\micro\meter}) &
        (\si{\pico\watt}) &
        (\si{\giga\hertz}) &
        (\si{\giga\hertz}) \\
        \midrule
        \\
        \textbf{Bulk Liquid} & \(2\pi\cdot\)\num{90(10)} & \num{10} & \num{1150} & \num{1.59} & \num{13(2)} & \(\sim\)\num{7.2} & \(2\pi\cdot\)\num{2.54(3)} & \(2\pi\cdot\)\num{0.575} \\
        \\
        \bottomrule
        \\
    \end{tabular}
    \caption[Material parameters for bulk liquid \ce{CS2} relevant to observing Brillouin traveling-wave modes and Raman standing-wave modes.]{Material parameters for bulk liquid \ce{CS2} relevant to observing Brillouin traveling-wave modes and Raman standing-wave modes, obtained from published values as well as our own observations shown in Figures~\ref{fig:Raman:1cmCS2}, \ref{fig:Raman:4mmCS2}, \ref{fig:Raman:1mmCS2}, and \ref{fig:Raman:100umCS2}. Here, \(\Gamma_{\rm B}\) is the angular Brillouin linewidth (phonon dissipation rate) and the inverse of phonon lifetime (\(\tau = \Gamma_{\rm B}^{-1}\)), \(v_{\rm s,\,long}\) is the longitudinal sound speed, \(n\) is the refractive index, \(l_{\rm coh}\) is the phonon coherence length (mean travel distance), and \(P_{\rm CoBS}\) is the scattered power for the \ac{CoBS} process, reported here for \(L_{\rm coh}/2 =\) \SI{6.5}{\micro\meter} \ce{CS2}, and scales with \(L^{2}\) (Equation~\ref{eq:Raman:ScatteredPowerPhi}). Finally, \(\Omega_{\rm B}\) is the angular Brillouin frequency shift (Equation~\ref{eq:Raman:f_B}), and \(\Omega_{\rm R,\,\SI{10}{\micro\meter}}\) is the first harmonic (\(n=1\)) of the fundamental \(L_{0}\) Raman-like mode for \(L=\)\SI{1}{\micro\meter} (Equation~\ref{eq:Raman:f_R}).}
    \label{tab:Raman:CS2}
\end{table}

\begin{figure}[t]
    \centering
    \begin{subfigure}[b]{0.49\textwidth}
        \centering
        \includegraphics[width=\textwidth]{figs/4-Raman/1cmCS2.jpeg}
        \caption{}
        \label{fig:Raman:1cmCS2pic}
    \end{subfigure}
    \hfill
    \begin{subfigure}[b]{0.49\textwidth}
        \centering
        \includegraphics[width=\textwidth]{figs/4-Raman/4mmCS2.jpg}
        \caption{}
        \label{fig:Raman:4mmCS2pic}
    \end{subfigure}
    \caption{\SI{1}{\centi\meter} (\ref{fig:Raman:1cmCS2}) and \SI{4}{\milli\meter} (\ref{fig:Raman:4mmCS2}) liquid \ce{CS2}.}
    \label{fig:Raman:CS2Cuvet}
\end{figure}

\begin{figure}[t]
  \centering
  \hspace{-2em}\includegraphics[width=.85\textwidth]{figs/4-Raman/CoBS Measurement: 1 cm CS2.png}
  \caption[\ac{CoBS} measurement of \SI{1}{\centi\meter} liquid \ce{CS2}.]{Observed background-subtracted spectrum obtained through a \ac{CoBS} measurement of \SI{1}{\centi\meter} of liquid \ce{CS2} (pictured in Figure~\ref{fig:Raman:1cmCS2}), with 1\(\sigma\) uncertainties smaller than the data point markers. The spectrum shows strong Fano asymmetry (see Section ~\ref{Appendix:Fano} in Appendix~\ref{appendix: CoBS}), indicating interference of a weak signal with the broad background continnuum. This is an older measurement, taken prior to orders of magnitude sensitivity improvements, and the Fano effects seen here indicate this was near the sensitivity limit of the instrument at the time.}
  \label{fig:Raman:1cmCS2}
\end{figure}

\begin{figure}[t]
  \centering
  \hspace{-2em}\includegraphics[width=.85\textwidth]{figs/4-Raman/CoBS Measurement: 4 mm CS2.png}
  \caption[\ac{CoBS} measurement of \SI{4}{\milli\meter} liquid \ce{CS2}.]{Observed background-subtracted spectrum obtained through a \ac{CoBS} measurement of \SI{4}{\milli\meter} of liquid \ce{CS2} (pictured in Figure~\ref{fig:Raman:4mmCS2}), with 1\(\sigma\) uncertainties smaller than the data point markers. This is an older measurement, taken prior to orders of magnitude sensitivity improvements, and the Fano effects seen here indicate this was near the sensitivity limit of the instrument at the date of this measurement.}
  \label{fig:Raman:4mmCS2}
\end{figure}

\subsection{Tellurium Dioxide Thin Film}
\label{subsec:Raman:Target:TeO2}

\begin{itemize}
  \item Gibbs collab
  \item deposit Te, oxidize into TeO2
  \item table of relevant TeO2 parameters
\end{itemize}

\begin{table}[h]
    \centering
    \begin{tabular}{c c c c c c c c c}
        \toprule
        \textbf{\ce{TeO2}} &
        \(\mathbf{\Gamma_{\rm \textbf{B}}}\) \cite{renninger2018bulk} &
        \(\mathbf{\tau}\) &
        \(\mathbf{v_{\rm \textbf{s,\,long}}}\) \cite{uchida1969elastic, schweppe1970elastic, ohmachi1972acoustic, peercy1975temperature, fleury2018non, harris1991multichannel} &
        \(\mathbf{n}\) \cite{uchida1971optical} &
        \(\mathbf{L_{\rm \textbf{coh}}}\) &
        \(\mathbf{P_{\rm \textbf{CoBS,\,\(L_{\rm coh}/2\)}}}\) &
        \(\mathbf{\Omega_{\rm \textbf{B}}}\) &
        \(\mathbf{\Omega_{\rm \textbf{R,\,\SI{1}{\micro\meter}}}}\) \\
        &
        (\si{\mega\hertz}) &
        (\si{\nano\second}) &
        (\si{\meter\per\second}) &
        &
        (\si{\micro\meter}) &
        (\si{\pico\watt}) &
        (\si{\giga\hertz}) &
        (\si{\giga\hertz}) \\
        \midrule
        \\
        \textbf{Crystal} & \(2\pi\cdot\)\num{10} & \num{100} & \num{4260} & \num{2.2} & \num{430} & \(\sim\)\num{3.5e3} & \(2\pi\cdot\)\num{12.1} & \(2\pi\cdot\)\num{2.13} \\
        \\
        \textbf{Thin Film} & \(2\pi\cdot\)\num{50(10)} & \num{20} & \(\sim\)\num{4150} & \num{2.27} & \num{83} & \(\sim\)\num{137} & \(2\pi\cdot\)\num{12.2} & \(2\pi\cdot\)\num{2.08} \\
        \\
        \bottomrule
        \\
    \end{tabular}
    \caption[Material parameters for \ce{TeO2} relevant to observing Brillouin traveling-wave modes and Raman standing-wave modes.]{Material parameters for \ce{TeO2} relevant to observing Brillouin traveling-wave modes and Raman standing-wave modes. The first row gives values for crystalline \ce{TeO2} \cite{renninger2018bulk, uchida1971optical}, while the second row reports measured values of our \SI{1}{\micro\meter} \ce{TeO2} thin film obtained from Figures~\ref{fig:Raman:1umTeO2} and \ref{fig:Raman:1umTeO2_combined}. Here, \(\Gamma_{\rm B}\) is the angular Brillouin linewidth (phonon dissipation rate) and the inverse of phonon lifetime (\(\tau = \Gamma_{\rm B}^{-1}\)), \(v_{\rm s,\,long}\) is the longitudinal sound speed, and \(n\) is the refractive index. We take the refractive index for an amorphous thin film of \ce{TeO2} as the average of the crystalline-\ce{TeO2} ordinary (\(n(o) = 2.2\)) and extraordinary (\(n(e) = 2.34\)) refractive indices (\(n_{\rm avg} \approx 2.27\)), giving an expectedly reduced longitudinal sound speed \(v_{\rm s,\,film} \approx\)\SI{4150}{\meter\per\second} in the imperfect lattice of the \ac{PVD}-deposited film. \(l_{\rm coh}\) is the phonon coherence length (mean travel distance), scattered power for a \ac{CoBS} process (\(P_{\rm CoBS}\)), reported here for \(L_{\rm coh,\,crystalline}/2 =\) \SI{215}{\micro\meter} and \(L_{\rm coh,\,thin\,film}/2 =\) \SI{42.5}{\micro\meter} \ce{TeO2}, scales with \(L^{2}\) (Equation~\ref{eq:Raman:ScatteredPowerPhi}). \(P_{\rm CoBS}\) improves slightly in the thin film due to the higher refractive index, but worstens significantly by faster dissipation \(\Gamma_{\rm B}\). Finally, \(\Omega_{\rm B}\) is the angular Brillouin frequency shift (Equation~\ref{eq:Raman:f_B}), and \(\Omega_{\rm R,\,\SI{1}{\micro\meter}}\) is the first harmonic (\(n=1\)) of the fundamental \(L_{0}\) Raman-like mode for \(L=\)\SI{1}{\micro\meter} (Equation~\ref{eq:Raman:f_R}).}
    \label{tab:Raman:TeO2}
\end{table}

\begin{figure}[t]
  \centering
  \includegraphics[width=.85\textwidth]{figs/4-Raman/slide-with-TeO2-film-on-substrate.jpeg}
  \caption{Glass slide with a thin film of \ce{TeO2} deposited via \ac{PVD}.}
  \label{fig:Raman:TeO2slide}
\end{figure}

\begin{figure}[t]
  \centering
  \hspace{-2em}\includegraphics[width=.85\textwidth]{figs/4-Raman/CoBS Measurement: 1 μm TeO2 wide.png}
  \caption{Observed background-subtracted spectrum obtained through a \ac{CoBS} measurement of \SI{1}{\micro\meter} of \ce{TeO2} captured at maximum operating optical powers \(P_{\rm P}P_{\rm S}P_{\rm Pr} \sim \SI{0.25}{\cubic\watt}\), with 1\(\sigma\) uncertainties.}4
  \label{fig:Raman:1umTeO2}
\end{figure}

\begin{figure}[t]
  \centering
  \begin{subfigure}[b]{0.85\textwidth}
    \centering
    \hspace{-2em}\includegraphics[width=\textwidth]{figs/4-Raman/CoBS Measurement: 1 μm TeO2.png}
    \caption{Raw resolution.}
    \label{fig:Raman:1umTeO2Raw}
  \end{subfigure}

  \vspace{1em}

  \begin{subfigure}[b]{0.49\textwidth}
    \centering
    \includegraphics[width=\textwidth]{figs/4-Raman/CoBS Measurement: 1 μm TeO2 5 MHz bin.png}
    \caption{\SI{5}{\mega\hertz} binning.}
    \label{fig:Raman:1umTeO25MHzBin}
  \end{subfigure}
  \hfill
  \begin{subfigure}[b]{0.49\textwidth}
    \centering
    \includegraphics[width=\textwidth]{figs/4-Raman/CoBS Measurement: 1 μm TeO2 10 MHz bin.png}
    \caption{\SI{10}{\mega\hertz} binning.}
    \label{fig:Raman:1umTeO10MHzBin}
  \end{subfigure}

  \caption{Observed background-subtracted spectra obtained through a \ac{CoBS} measurement of \SI{1}{\micro\meter} of \ce{TeO2} captured at maximum operating optical powers \(P_{\rm P}P_{\rm S}P_{\rm Pr} \sim \SI{0.25}{\cubic\watt}\), with 1\(\sigma\) uncertainties. Each panel shows the same data set under different binning resolutions.}
  \label{fig:Raman:1umTeO2_combined}
\end{figure}

\begin{figure}[t]
  \centering
  \hspace{-2em}\includegraphics[width=\textwidth]{figs/4-Raman/CoBS Measurement: 500 nm TeO2.png}
  \caption{Observed background-subtracted spectrum obtained through a \ac{CoBS} measurement of \SI{500}{\nano\meter} of \ce{TeO2} captured at maximum operating optical powers \(P_{\rm P}P_{\rm S}P_{\rm Pr} \sim \SI{0.25}{\cubic\watt}\), with 1\(\sigma\) uncertainties.}
  \label{fig:Raman:500nmTeO2}
\end{figure}

\subsection{Tellurium Thin Film}
\label{subsec:Raman:Target:Te}

\begin{itemize}
  \item Gibbs collab and CINT collab
  \item Se adhesion layer (sound speed \cite{kozhevnikov2007sound})
  \item oxidizes to TeO2
  \item table of relevant Te parameters
\end{itemize}

\begin{table}[h]
    \centering
    \begin{tabular}{c c c c c c c c c}
        \toprule
        \textbf{\ce{Te}} &
        \(\mathbf{\Gamma_{\rm \textbf{B}}}\) \cite{balakshii2008investigation, lin2016tellurium, voloshinov2017optic, khorkin2020acousto, voloshinov2008acousto} &
        \(\mathbf{\tau}\) &
        \(\mathbf{v_{\rm \textbf{s,\,long}}}\) \cite{balakshii2008investigation, lin2016tellurium, voloshinov2017optic, khorkin2020acousto, voloshinov2008acousto} &
        \(\mathbf{n}\) \cite{ciesielski2018permittivity, hartig1954infrared} &
        \(\mathbf{L_{\rm \textbf{coh}}}\) &
        \(\mathbf{P_{\rm \textbf{CoBS,\,\(\mathbf{L_{\rm coh}}/2\)}}}\) &
        \(\mathbf{\Omega_{\rm \textbf{B}}}\) &
        \(\mathbf{\Omega_{\rm \textbf{R,\,\textbf{\SI{1}{\micro\meter}}}}}\) \\
        &
        (\si{\mega\hertz}) &
        (\si{\nano\second}) &
        (\si{\meter\per\second}) &
        &
        (\si{\micro\meter}) &
        (\si{\pico\watt}) &
        (\si{\giga\hertz}) &
        (\si{\giga\hertz}) \\
        \midrule
        \\
        \textbf{Bulk} & \(\sim2\pi\cdot\)\num{10} & \num{100} & \(\sim\)\num{2610} & \num{4.58} & \num{261} & \(\sim\)\num{13e-3} & \(2\pi\cdot\)\num{15.4} & \(2\pi\cdot\)\num{1.31} \\
        \\
        \bottomrule
        \\
    \end{tabular}
    \caption[Material parameters for \ce{Te} relevant to observing Brillouin traveling-wave modes and Raman standing-wave modes.]{Material parameters for \ce{Te} relevant to observing Brillouin traveling-wave modes and Raman standing-wave modes, obtained from published values for bulk \ce{Te}. Here, \(\Gamma_{\rm B}\) is the angular Brillouin linewidth (phonon dissipation rate) and the inverse of phonon lifetime (\(\tau = \Gamma_{\rm B}^{-1}\)), \(v_{\rm s,\,long}\) is the longitudinal sound speed, \(n\) is the refractive index, \(l_{\rm coh}\) is the phonon coherence length (mean travel distance), and \(P_{\rm CoBS}\) is the scattered power for the \ac{CoBS} process, reported here for \SI{1}{\micro\meter} \ce{Te}, and scales with \(L^{2}\) (Equation~\ref{eq:Raman:ScatteredPowerPhi}). Finally, \(\Omega_{\rm B}\) is the angular Brillouin frequency shift (Equation~\ref{eq:Raman:f_B}), and \(\Omega_{\rm R,\,\SI{1}{\micro\meter}}\) is the first harmonic (\(n=1\)) of the fundamental \(L_{0}\) Raman-like mode for \(L=\)\SI{1}{\micro\meter} (Equation~\ref{eq:Raman:f_R}). While \ce{CS2} and \ce{TeO2} are transparent at \SI{1.55}{\micro\meter}, \ce{Te} is absorptive here. However, transmission becomes meaningful through thin films, with \SI{1}{\micro\meter} of deposited \ce{Te} permitting 11.4\% and \SI{500}{\nano\meter} permitting 29\% of \SI{1.55}{\micro\meter} light to transmit. \cite{ciesielski2018permittivity} This extra \(\sim\)90\% loss (for each of the three optical fields) has been accounted for in the scattered power value for the \ac{CoBS} process listed in the table.}
    \label{tab:Raman:Te}
\end{table}

\begin{figure}[t]
  \centering
  \includegraphics[width=.8\textwidth]{figs/4-Raman/AcousticImpedance.png}
  \caption{Acoustic impedance.}
  \label{fig:Raman:AcousticImpedance}
\end{figure}

\subsection{Carbon Disulfide Micrometer Cell}
\label{subsec:Raman:Target:CS2Cells}

\begin{itemize}
  \item table of relevant CS2 parameters
  \item cells, 1 W amp, bubble test
\end{itemize}

\begin{figure}[t]
    \centering
    \begin{subfigure}[b]{0.3\textwidth}
        \centering
        \includegraphics[width=\textwidth]{figs/4-Raman/1mmCS2.jpg}
        \label{fig:Raman:1mmCS2pic}
    \end{subfigure}
    \hfill
    \begin{subfigure}[b]{0.3\textwidth}
        \centering
        \includegraphics[width=\textwidth]{figs/4-Raman/100umCS2.jpg}
        \label{fig:Raman:100umCS2pic}
    \end{subfigure}
    \hfill
    \begin{subfigure}[b]{0.3\textwidth}
        \centering
        \includegraphics[width=\textwidth]{figs/4-Raman/10umCS2.jpg}
        \label{fig:Raman:10umCS2pic}
    \end{subfigure}
    %
    \caption{Three \ce{CS2} cells of different path lengths (\SI{1}{\milli\meter}, \SI{100}{\micro\meter}, and \SI{10}{\micro\meter}) secured in the beam path of the \acl{CoBS}.}
    \label{fig:Raman:CS2Comparison}
\end{figure}

\begin{figure}[t]
  \centering
  \hspace{-2em}\includegraphics[width=.85\textwidth]{figs/4-Raman/CoBS Measurement: 1 mm CS2.png}
  \caption{Observed background-subtracted spectra obtained through a \ac{CoBS} measurement of \SI{1}{\milli\meter} of liquid \ce{CS2} (pictured in Figure~\ref{fig:Raman:1mmCS2}) captured at maximum operating optical powers \(P_{\rm P}P_{\rm S}P_{\rm Pr} \sim\) \SI{0.25}{\cubic\watt}, with 1\(\sigma\) uncertainties smaller than the data point markers.}
  \label{fig:Raman:1mmCS2}
\end{figure}

\begin{figure}[t]
  \centering
  \hspace{-2em}\includegraphics[width=.85\textwidth]{figs/4-Raman/CoBS Measurement: 100 μm CS2.png}
  \caption{Observed background-subtracted spectra obtained through a \ac{CoBS} measurement of \SI{100}{\micro\meter} of liquid \ce{CS2} (pictured in Figure~\ref{fig:Raman:100umCS2}) captured at maximum operating optical powers \(P_{\rm P}P_{\rm S}P_{\rm Pr} \sim\) \SI{0.25}{\cubic\watt}, with 1\(\sigma\) uncertainties smaller than the data point markers.}
  \label{fig:Raman:100umCS2}
\end{figure}

\begin{figure}[t]
  \centering
  \hspace{-2em}\includegraphics[width=.85\textwidth]{figs/4-Raman/CoBS Measurement: 10 μm CS2.png}
  \caption{Observed background-subtracted spectra obtained through a \ac{CoBS} measurement of \SI{10}{\micro\meter} of liquid \ce{CS2} (pictured in Figure~\ref{fig:Raman:10umCS2}) captured at maximum operating optical powers \(P_{\rm P}P_{\rm S}P_{\rm Pr} \sim\) \SI{0.25}{\cubic\watt}, with 1\(\sigma\) uncertainties smaller than the data point markers.}
  \label{fig:Raman:10umCS2}
\end{figure}

\subsection{Suspended Silica Rib Waveguide}
\label{subsec:Raman:Target:Waveguide}

\begin{itemize}
  \item BYU collab
  \item If we can couple chip waveguide into CoBS fiber-chip-fiber, then we have access to a playground of materials and geometries
  \item initial test took 9 months to learn and measure
\end{itemize}

\begin{figure}[t]
  \centering
  \hspace{-2em}\includegraphics[width=.85\textwidth]{figs/4-Raman/CoBS Measurement: 1 cm Suspended Silica Rib Waveguide.png}
  \caption{\ac{CoBS} measurement of a suspended silica rib waveguide for a backward scattering process. A subsequent measurement with the chip waveguide removed is also plotted. The spectral response of the \ac{SMF-28} which comprises much of the sample stage of the instrument is visible without the chip, however a notable signal from the waveguide can be seen above the background. In obtaining the spectra, five repeated measurements of both the signal and background (probe off) were collected at a \SI{100}{\hertz} \ac{RBW}, dwelling for \SI{1}{\second} at each \SI{5}{\mega\hertz} frequency step. Plotted is the resulting background-subtracted spectrum. Uncertainties represent 1\(\sigma\) standard error of the mean.}
  \label{fig:Raman:August2024chipnochip}
\end{figure}

\subsection{Elastically-Suspended Photonic-Phononic Waveguide}
\label{subsec:Raman:Target:WigglyWaveguide}

Figure~\ref{fig:Raman:wigglyCoBSspectra} plots the spectra collected from a \ac{CoBS} measurement of backward Brillouin scattering in the elastically-suspended silica rib of the photonic-phononic waveguide. In this section, we present theoretical estimates of the key mechanical modes and optomechanical couplings predicted for the elastically-suspended photonic-phononic waveguide under development. The structure under consideration consists of a polymeric (SU-8) membrane that is tensioned during high-temperature curing and supports a rib waveguide above it. This combined membrane-and-rib geometry gives rise to multiple distinct mechanical resonances that can be excited and probed optically via the \ac{CoBS} instrument operating in the forwards scattering configuration. We summarize below the principal modes, referred to as ``drumhead'' modes for the membrane and ``breathing'' modes for the rib, and estimate rough frequency ranges in which they are expected to appear.

\begin{figure}[t]
  \centering
  \hspace{-2em}\includegraphics[width=.85\textwidth]{figs/4-Raman/CoBS: 1 cm Elastically-Suspended Photonic-Phononic Waveguide.png}
  \caption{\ac{CoBS} measurement of the elastically-suspended photonic-phononic waveguide for a backward scattering process. A subsequent measurement with the chip waveguide removed reveals a similar spectral profile, as expected from the identical resonant frequency response of the \ac{SMF-28} which comprises much of the apparatus, especially in the sample region of the instrument where all three optical powers overlap. In obtaining the spectra, five repeated measurements of both the signal and background (probe off) were collected at a \SI{100}{\hertz} \ac{RBW}, dwelling for \SI{1}{\second} at each \SI{5}{\mega\hertz} frequency step. Plotted is the resulting background-subtracted spectrum. Uncertainties represent 1\(\sigma\) standard error of the mean.}
  \label{fig:Raman:wigglyCoBSspectra}
\end{figure}

We first consider the rib portion of the waveguide, having characteristic lateral width and thickness of \SI{4}{\centi\meter} and \SI{6}{\centi\meter}, respectively. One can treat the smallest cross-sectional dimension, denoted \(d\), as setting the approximate half-wavelength condition for its fundamental breathing mode. The simplest estimate for such a half-wavelength mechanical resonance is given by

\begin{equation}
f_{\rm rib,breathe} \approx \frac{v_{\rm s}}{2d},
\end{equation}
\\
where \(v_{\rm s}\) is the speed of sound in the silica rib and \(d\) is the lateral width. Taking the rib to be \SI{6}{\micro\meter} wide and assuming a speed of sound in the range \SI{5}{\kilo\meter\per\second} to \SI{6}{\kilo\meter\per\second}, this places the fundamental rib breathing mode near several hundred \si{\mega\hertz} (estimates range from \SI{400}{\mega\hertz} to \SI{500}{\mega\hertz}). Higher-order modes can arise (integer multiples of the half-wavelength condition), and so in practice one expects a series of possible breathing modes in the 100s of \si{\mega\hertz} to low-\si{\giga\hertz} range.

In principle, when optically excited via the \ac{CoBS} process, the rib breathing mode can couple to and drive displacement of the underlying membrane. Conversely, membrane motion at or near the rib’s breathing-frequency range can stimulate the rib's motion. Indeed, the waveguide design permits either the drumhead mode of the membrane or the rib breathing mode to be excited off-resonance and potentially induce mechanical vibration in the other. Beneath the rib, the polymer membrane is suspended over an open region which has been etched away, and is anchored on either side. We thus expect a classical drumhead family of modes in which the membrane oscillates out-of-plane. The tension provided by the fabrication curing process indicates that the membrane is tension-dominated. For a square membrane of length and width \(L\), under tension \(T\) and with mass density \(\rho\) and thickness \(h\), the fundamental frequency (1,1) can be approximated by

\begin{equation}
  f_{\rm membrane,\,drumhead\,(1,1)} \approx \frac{v_{\rm s}}{\sqrt{2}L},
  \label{eq:Raman:SquareMembrane}
\end{equation}
\\
where \(v_{\rm s}\) is the \emph{transverse} sound speed of the membrane, given by

\begin{equation}
  v_{\rm s} = \sqrt{\frac{T}{\rho h}} = \sqrt{\frac{\sigma h}{\rho h}} = \sqrt{\frac{\sigma}{\rho}},
  \label{eq:Raman:TransverseSoundSpeed}
\end{equation}
\\
where \(\sigma\) is the stress (\si{\pascal}) corresponding to the tension per membrane thickness (\(\sigma = T/h\)).

Modeling the polymer membrane as a square membrane, we can estimate the fundamental membrane drumhead mode using a mean density of SU8 (\(\rho=\)\SI{1190}{\kilo\gram\per\cubic\meter}) \cite{roch2003fabrication} and needle profilometer tension measurements performed on the membrane. Figure~\ref{fig:Raman:profilometer} plots deflection for various stylus force values applied to the membrane. The inverse of the slope of these data (taken to be linear, \(m\sim\)\SI{300}{\newton\per\meter}) allows for the membrane's approximate tension to be found via \(T=\frac{L}{4}m\sim\)\SI{2}{\milli\newton\per\meter}. For a \SI{2}{\micro\meter}-thickness with Equation~\ref{eq:Raman:TransverseSoundSpeed}, this gives an approximate transverse sound speed of a very low \(v_{\rm s}\sim\)\SI{1}{\meter\per\second}. Inserting this into Equation~\ref{eq:Raman:SquareMembrane} with length \(L\approx\)\SI{25}{\micro\meter}, we anticipate the fundamental drumhead mode resonant frequency of the SU8 membrane to lie around \SI{30}{\kilo\hertz}. Meanwhile, the rib breathing mode are predicted to lie around the mid-100s of \si{\mega\hertz}. Given this 4-order-of-magnitude discrepancy in resonant frequencies, the two mechanical modes are not expected to exhibit mechanical coupling, even when accounting for the possibility of off-resonant driving. To have the two resonances approach each other, one might alter the fabrication parameters to try significantly increasing the resulting tension of the membrane, perhaps by experimenting with different bake times or temperatures and membrane thicknesses. Alternatively, or additionally, shortening the length \(L\) across which the membrane spans from clamp-to-clamp would minimally nudge its resonant frequency higher.

\begin{figure}[t]
  \centering
  \includegraphics[width=0.7\textwidth]{figs/4-Raman/profilometer.png}
  \caption{Needle profilometer force-deflection measurements (illustrated above) performed on the suspended polymer membrane of the photonic-phononic waveguide. The inverse slope of the data (taken to be linear) is proportional to the tension of the membrane (\(m\propto T\)). Using \(m\sim\)\SI{300}{\newton\per\meter}, we estimate the tension of the membrane to be \(T\sim\)\SI{2}{\milli\newton\per\meter}, giving a transverse sound speed \(v_{\rm s}\sim\)\SI{1}{\meter\per\second} via Equation~\ref{eq:Raman:TransverseSoundSpeed}. Figure courtesy of Adams et al., \textit{manuscript in prep}.}
  \label{fig:Raman:profilometer}
\end{figure}

\begin{figure}[t]
  \centering
  \includegraphics[width=\textwidth]{figs/4-Raman/StartBigApproachSmall.png}
  \caption{Start big, approach small.}
  \label{fig:StartBigApproachSmall}
\end{figure}

\begin{figure}[t]
  \centering
  \includegraphics[width=\textwidth]{figs/4-Raman/HowWouldRamanModesAppear.png}
  \caption{How would Raman modes appear.}
  \label{fig:HowWouldRamanModesAppear}
\end{figure}

%--------------------------------------------------------------------%

\section{Discussion}
\label{sec:Raman:Discussion}

\subsection{Pathways to Brillouin-Induced Raman Modes}
\label{subsec:Raman:Pathways}

Ideal platforms by category
\begin{itemize}
  \item waveguide - long TeO2 rib, evenly spaced square holes
  \item TeO2 thin film/crystal - dissolve only small area of substrate for beam spot
  \item CS2 cell - 5um
  \item Fiber - notched, acoustic fiber Bragg grating
\end{itemize}

\subsection{Conclusion}
\label{subsec:Raman:Conclusion}

Connection to Dissertation Theme
\begin{itemize}
  \item Relate back to the broader aims of controlling phonons at room temperature, highlighting how these efforts extend exploration of optomechanical interactions.
\end{itemize}

\clearpage
\thispagestyle{empty}
\null
\newpage
