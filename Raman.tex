\setcounter{rownumber}{0}
\singlespacing
\chapter{Manuscript III: Brillouin-induced Raman modes}
\label{ch:Raman}
\acresetall

%  Copy this file for each main chapter, make sure to add it to main.tex

% Example author list with footnote style affiliations
%
% Christian J. Tai Udovicic\footnote{\label{spweather:nau}\ac{NAU}, \ac{DAPS}, PO Box 6010, Flagstaff, AZ 86011, USA}, Emily S. Costello\footnote{University of Hawai'i at Manoa, Honolulu, HI, USA}, Rebecca R. Ghent\footnote{Planetary Science Institute, Tucson, AZ, USA}, Christopher S. Edwards$^\mathrm{\ref{spweather:nau}}$

%  Extra copyright disclaimer to be safe
%
% \textit{This is the Accepted Manuscript version of an article accepted for publication in \ac{GRL}. Wiley Inc is not responsible for any errors or omissions in this version of the manuscript or any version derived from it. The Version of Record is available online at} \url{https://doi.org/10.1029/2020GL092198}\textit{.}

\doublespacing

%%%%%%%%%%%%%%%%%%%%%%%%%%%%%%%%%%%%%%%%%%%%%%%%%%%%%%%%%%%%%%%%%%%%
%  Chapter contents here

\section{Abstract}
\lipsum[2]


\section{Introduction}
\lipsum[2]

\begin{table}[htb]
\caption{Table caption.}
\centering
\begin{tabular}{l l c l}
\hline
& Parameter & Value & Description  \\
\hline
\multirow{5}{6em}{Lookup Variables}
 & lat  & -85$\degree$--85$\degree$ & Latitude (35 bins in 5$\degree$ increments)  \\
 & ALBEDO  & 0.05--0.225 & Bolometric albedo (6 bins in 0.035 increments)  \\
 & SLOPE  & 0$\degree$--90$\degree$ & Surface slope (19 bins in 5$\degree$ increments)   \\
 & SLOAZI  & 0$\degree$--360$\degree$ & Surface azimuth (19 bins in 20$\degree$ increments)   \\
 & DELLS  & 4$\degree$ & $L_s$ step size (90 bins spanning 0$\degree$--360$\degree$) \\
\hline
\multirow{8}{6em}{Thermal Parameters}
 & EMISS  & 0.96 & Emissivity  \\
 & thick  & 0.05 & Upper layer thickness [m] \\
 & DENSITY  & 1100 & Upper layer density [kg/m$^3$]  \\
 & DENS2  & 1800 & Lower layer density [kg/m$^3$]  \\
 & lbound  & 18 & Interior heat flow [mW/m$^2$]   \\
 & \multirow{3}{*}{PhotoFunc}  & \multirow{3}{*}{0.045/albedo} & \multirow{3}{20em}{Photometric function (Keihm-style)} \\
 & & & \\
 & & & \\
\hline
\multirow{12}{6em}{Temperature-dependent parameters}
 & SphUp0/SphLo0  & 602.88098583 & \multirow{4}{20em}{Specific heat capacity expressed as 4th-order polynomial ($\rm c0 + c1 \cdot T + c2 \cdot T^2 + c3 \cdot T^3$)} \\
 & SphUp1/SphLo1  & 235.98988249 &  \\
 & SphUp2/SphLo2  & -29.59742178 &  \\
 & SphUp3/SphLo3  & -3.78707193  & \\
 \\
 & ConUp0  & 0.00133644 &  \multirow{4}{20em}{Upper layer conductivity expressed as 4th-order polynomial ($\rm c0 + c1 \cdot T + c2 \cdot T^2 + c3 \cdot T^3$)} \\
 & ConUp1  & 0.00073150 &  \\
 & ConUp2  & 0.00033250 &  \\
 & ConUp3  & 0.00005038 &  \\
 \\
 & ConLo0  & 0.00634807 &  \multirow{4}{20em}{Lower layer conductivity expressed as 4th-order polynomial ($\rm c0 + c1 \cdot T + c2 \cdot T^2 + c3 \cdot T^3$)} \\
 & ConLo1  & 0.00347464 &  \\
 & ConLo2  & 0.00157938 &  \\
 & ConLo3  & 0.00023930 &  \\
\hline
\multirow{8}{6em}{Model Setup Parameters}
 & body  & Moon & Target body  \\
 & k\_style  & Moon & Conductivity style (Moon for airless bodies)  \\
 & LKofT  & T & Temperature-dependent conductivity  \\
 & FLAY  & 0.01 & First layer thickness [m]  \\
 & RLAY  & 1.3 & Layer thickness multiplier    \\
 & N1  & 26 & Number of layers  \\
 & N24  & 288 & Timesteps per day (5 min steps)  \\
 & DJUL  & 0 & Start date  \\
\hline
\end{tabular}
\label{stab:parametersIII}
\end{table}

\lipsum[2]
