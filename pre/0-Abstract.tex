\phantomsection
\addcontentsline{toc}{chapter}{Abstract}

\begin{center}
    \large
    ABSTRACT

    \large
    \dissertationTitle{}

    \large
    \vspace{2mm}
    JOEL N. JOHNSON

\end{center}

% Abstract text max 350 words recommended
\noindent

This dissertation explores novel ways in which light and sound interact in optical waveguides to enable laser cooling of traveling-wave phonons and to enhance phonon spectroscopy. First, we establish how anti-Stokes Brillouin scattering can be harnessed to cool acoustic excitations in a continuous fiber, extending laser cooling concepts beyond discrete cavity optomechanical systems. By using a \ac{LCOF} filled with \ce{CS2}, we achieve significant reduction in the phonon population through spontaneous anti-Stokes processes. This demonstration marks the first instance of in-fiber traveling-wave phonon cooling without requiring nanophotonic resonators or specialized microstructures.

Second, we introduce a Coherently Stimulated Brillouin Spectrometer (\acs{CoBS}) that employs a four-wave mixing scheme comprising pump, probe, driven Stokes, and observed signal fields to dramatically improve sensitivity, especially for short, low-gain samples. By relaxing traditional phase-matching constraints, the CoBS approach enables a \(10^{6}\) increase in scattered power over millimeter- to centimeter-scale lengths and demonstrates sub-\SI{10}{\femto\watt} sensitivity. We validate its utility with measurements in both high- and low-gain media, opening new possibilities for ultrathin-film and microfluidic analyses.

Lastly, this dissertation investigates the theoretical and experimental conditions under which Brillouin-driven Raman-like modes might be observed. These Brillouin-induced Raman modes could provide a bridge between conventional Raman spectroscopy and the acousto-optic phenomena characteristic of traveling-wave phonons. While not yet fully realized, the groundwork here points to future avenues in which engineering the geometric and material properties of optical waveguides can yield both net phonon cooling and advanced hybrid phononic devices. Collectively, these results broaden our understanding of optomechanical coupling in fibers and pave the way for low-noise photonic systems, quantum acoustic memory elements, and highly sensitive, broadband spectroscopic tools.
