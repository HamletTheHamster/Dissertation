\setcounter{rownumber}{0}
\singlespacing
\chapter{Manuscript II: A coherently stimulated phonon spectrometer}
\label{ch:CABS}
\acresetall

%  Copy this file for each main chapter, make sure to add it to main.tex

% Example author list with footnote style affiliations
%
Joel N. Johnson\footnote{\label{CABS-NAU}
Department of Applied Physics and Materials Science, Northern Arizona University, Flagstaff, AZ 86011, USA
}$^,$\footnote{\label{CABS-MIRA}
Center for Materials Interfaces in Research and Applications, Flagstaff, AZ 86011, USA
},
Nils T. Otterstrom\footnote{\label{CABS-Sandia}
Sandia National Laboratory, 1515 Eubank Blvd SE, Albuquerque, NM 87123, USA
},
Peter T. Rakich\footnote{\label{CABS-Yale}
Department of Applied Physics, Yale University, New Haven, CT 06520, USA
},
Ryan O. Behunin$^\mathrm{\ref{CABS-NAU}}$$^,$$^\mathrm{\ref{CABS-MIRA}}$

\hfill

%  Extra copyright disclaimer to be safe
%
\textit{This is the Accepted Manuscript version of an article accepted for publication in Nature Photonics. Wiley Inc is not responsible for any errors or omissions in this version of the manuscript or any version derived from it. The Version of Record is available online at} \url{https://doi.org/}\textit{.}

\doublespacing

%%%%%%%%%%%%%%%%%%%%%%%%%%%Nature Photonics Format Requirements%%%%%%%%%%%%%%%%%%%%%%%%%%%%%%%%%%%
%Article
%An Article is a substantial novel research study, with a complex story often involving several techniques or approaches.

%Format

%Main text – up to 3,000 words, excluding abstract, Methods, references and figure legends.
%Abstract – up to 200 words, unreferenced.
%Display items – up to 6 items (figures and/or tables).
%Article should be divided as follows:
%Introduction (without heading)
%Results
%Discussion
%Online Methods. ​
%Results and Methods should be divided by topical subheadings; the Discussion does not contain subheadings.
%References – as a guideline, we typically recommend up to 50.
%Articles include received/accepted dates.
%Articles may be accompanied by supplementary information.
%Articles are peer reviewed.

%%%%%%%%%%%%%%%%%%%%%%%%%%%%%%%%%%%%%%%%%%%%%%%%%%%%%%%%%%%%%%%%%%%%%%%%%%%%%%%%%%%%%%%%%%%%%%%%%%
%  Chapter contents here

\section{Abstract}
\label{sec:CABS:Abstract}
\lipsum[1]

\section{Introduction}
\label{sec:CABS:Introduction}
\lipsum[1]
State of brillouin microscopy
Applications and usefulness
Challenges: selection of backscattered signal
  conflated with Stokes field
  phase-matching requires probe wavelength to be exactly that of Stokes
Wouldn't it be nice if we could break free of strict phase-matching requirements, therefore perfectly isolating the signal
In this work

\section{Instrument Design} % online-only for Nature Photonics
\label{sec:CABS:Design}
\lipsum[1]

  \subsection{Design of instrument}
  \label{subsec:CABS:Design:Design}
  \lipsum[1]
  description of design
  figure: instrument
    apparatus design

  \subsection{Sensitivity Measurements}
  \label{subsec:CABS:Design:Sensitivity}
  \lipsum[1]

\section{Theory}
\label{sec:CABS:Theory}
\lipsum[1]

  \subsection{Coupled Wave Equations}
  \label{subsec:CABS:Theory:Coupled-Wave}
  \lipsum[1]
  full CABS theory arriving at scattered power

  \subsection{Phase-matching bandwidth}
  \label{subsec:CABS:Theory:Phase-Matching}
  \lipsum[1]
  phase-matching bandwidth theory

\section{Results}
\label{sec:CABS:Results}
\lipsum[1]

  \subsection{Fiber-Coupled: UHNA3}
  \label{subsec:CABS:Results:UHNA3}
  \lipsum[1]

  \subsection{Free-Space: \texorpdfstring{$CS_{2}$}{CS2}}
  \label{subsec:CABS:Results:CS2}
  \lipsum[1]
  figure: demonstration measurements
    1mm uhna3 fiber
    1mm CS2 bulk

    \begin{figure*}[t]
        \centering
        \includegraphics[width=\textwidth]{figs/4-CABS/CABS-100um CS2.png}
        \caption{CABS measurement of 100um of CS2.}
        \label{fig:CABS 100um CS2}
    \end{figure*}

    comparison to stimulated brillouin and spontaneous brillouin?

  \subsection{Phase-Matching in Small L Regime}
  \label{subsec:CABS:Results:Phase-Matching}
  \lipsum[1]
  figure: phase-matching
    peak vs pump-probe separation 1cm uhna3, CS2
    peak vs pump-probe separation 1mm uhna3, CS2

\section{Discussion}
\label{sec:CABS:Discussion}
\lipsum[1]

%The following are the topics I brainstormed out for possible appendices for
%the cabs paper as of 10-29-2024

%\section{Instrument Details}
%\subsection{List of Salient Components}
%\subsection{Laser Output: Whisper Mode vs. Dither Mode}
%\subsection{Lock-in Settings}

%\section{Experimental Techniques}
%\subsection{Background Subtraction: Probe Off, Sample In}
%\subsection{Background Subtraction: Probe On, Sample Out}
%\subsection{Polarization Control Limit}
%\subsection{Pump, Stokes, Probe Polarization Optimization}
%\subsection{Local Oscillator Polarization Optimization}
%\subsection{Lock-in Detector Settings}
%\subsection{Detection Bandwidth}

%\section{Pump, Stokes, and Probe Contribute Equally}

%\section{Comparison to SBS}
%\subsection{1 Centimeter UHNA3 Scattered Power}
%\subsection{100 Micrometers CS2 Scattered Power}

%\section{Measurement Theory}
%\subsection{Heterodyne Detection}
%\subsection{Lock-in Detection}

%\section{Sensitivity}
%\subsection{Sensitivity Measurements}
%\subsection{Current Sensitivity Limitors}
%\subsection{Ultimate Sensitivity Limitor: Shot Noise}

%\section{Alternative Configurations}
%\subsection{Mirrored Design}
%\subsection{Radial Acoustic Modes}
%\subsection{Shear Acoustic Modes}
%\subsection{Torsional Acoustic Modes}
%\subsection{Coherent Raman Spectroscopy}
