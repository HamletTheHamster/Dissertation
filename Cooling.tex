\setcounter{rownumber}{0}
\chapter{Laser cooling of traveling wave phonons in an optical fiber}
\label{ch:Cooling}
\acresetall

Joel N. Johnson\footnote{\label{Cooling-NAU}
Department of Applied Physics and Materials Science, Northern Arizona University, Flagstaff, AZ 86011, USA
}$^,$\footnote{\label{Cooling-MIRA}
Center for Materials Interfaces in Research and Applications, Flagstaff, AZ 86011, USA
},
Danielle R. Haverkamp$^\mathrm{\ref{Cooling-NAU}}$$^,$$^\mathrm{\ref{Cooling-MIRA}}$,
Yi-Hsin Ou\footnote{\label{Cooling-UofA}
College of Optical Sciences, University of Arizona, Tucson, AZ, USA
},
Khanh Kieu$^\mathrm{\ref{Cooling-UofA}}$,
Nils T. Otterstrom\footnote{\label{Cooling-Sandia}
Photonic and Phononic Microsystems, Sandia National Laboratories, Albuquerque, New Mexico, USA
},
Peter T. Rakich\footnote{\label{Cooling-Yale}
Department of Applied Physics, Yale University, New Haven, CT, USA
},
Ryan O. Behunin$^\mathrm{\ref{Cooling-NAU}}$$^,$$^\mathrm{\ref{Cooling-MIRA}}$

\hfill

\textit{This chapter elaborates on experiments and results related to the demonstration of optomechanical cooling of traveling wave phonons in optical fiber which have been published as an article by the same name in Physical Review Applied by Johnson et al. (2023) \cite{johnson2023laser}. Any discrepancies, omissions, or errors that may exist between the published paper and this dissertation chapter are the sole responsibility of the author, as the text, analyses, and interpretations herein represent an independent and original presentation of the work.}

%--------------------------------------------------------------------%

\section{Introduction}
\label{sec:Cooling:Introduction}

Materials above the ground state experience therodynamic variations such as temperature and density. These thermal fluctuations alter the optical properties of a given material, allowing a scattering process to occur (see Section \ref{sec:Introduction:Light-Scattering}). Spontaneous Brillouin scattering is the inelastic scattering of light with these thermal fluctuations within a material, facilitating an energy exchange between the optical and acoustic domains. While a given medium typically supports a multitude of thermally excited acoustic modes (including both transverse and longitudinal modes) we focus here specifically on longitudinally travelling acoustic waves to demonstrate cooling in a continuous (non-resonant) system.

[lit review, state of the art]

%--------------------------------------------------------------------%

\section{Optomechanical Cooling and Heating}
\label{sec:Cooling:Cooling-Heating}

Backward Brillouin scattering targets longitudinally traveling acoustic waves (or phonons) through two complementary processes---Stokes and anti-Stokes, illustrated in Figures \ref{fig:Cooling:StokesHeating} and \ref{fig:Cooling:anti-StokesCooling}, respectively. In the Stokes process, an incident photon of frequency \(\omega\) scatters with a \textit{retreating} phonon of frequency \(\Omega\) annihilating the \textit{photon} and creating both an additional phonon of frequency \(\Omega\) and a backscattered photon at the difference energy (\(\omega_{\mathrm{Stokes}} = \omega - \Omega\)). In this way, both energy and momentum are conserved. This can be visualized by an analogy, in which the incident light experiences a doppler \textit{down}-shift in frequeny as though the photon were reflected from a retreating mirror. Since this processes results in an increase in the phonon population within the respective longitudinal mode of the material, this process is referred to as optomechanical heating. The energy lost by the light is gained by the material in the form of mechanical vibrations.

\begin{figure}[t]
    \centering
    \begin{subfigure}[b]{0.49\textwidth}
        \centering
        \includegraphics[width=\textwidth]{figs/3-Cooling/StokesHeatingProcess.png}
        \caption{}
        \label{fig:Cooling:StokesHeating}
    \end{subfigure}
    \hfill
    \begin{subfigure}[b]{0.49\textwidth}
        \centering
        \includegraphics[width=\textwidth]{figs/3-Cooling/anti-StokesCoolingProcess.png}
        \caption{}
        \label{fig:Cooling:anti-StokesCooling}
    \end{subfigure}
    \caption{Illustration of optomechanical heating and cooling processes. Figure \ref{fig:Cooling:StokesHeating} shows an incident photon of frequency \(\omega\) scattering with a retreating phonon of frequency \(\Omega\), resulting in the annihilation of the incident photon and the creation of both an additional retreating phonon of frequency \(\Omega\) and a backwards propagating photon of reduced frequency and thereby energy (\(\omega_{\mathrm{Stokes}} = \omega - \Omega\)). Figure \ref{fig:Cooling:anti-StokesCooling} shows the inverse process, whereby an incident photon, \(\omega\), scatters with an approaching phonon, \(\Omega\), annihilating the incident photon and the phonon to produce a backwards propagating photon of increased frequency and thereby energy (\(\omega_{\mathrm{anti-Stokes}} = \omega + \Omega\)).}
    \label{fig:Cooling:StokesProcesses}
\end{figure}

The anti-Stokes process is the inverse process, whereby an \textit{approaching} phonon of frequency \(\Omega\) scatters with an incident photon of frequency \(\omega\), annihilating the \textit{phonon} and creating a backscattered photon at the addition energy (\(\omega_{\mathrm{anti-Stokes}} = \omega + \Omega\)). Both energy and momentum are again conserved, however in the anti-Stokes process, the incident light experiences a doppler \textit{up}-shift in frequency as if, to continue the analogy, the photon were reflected from an approaching mirror. For further intuition, one might consider an elastic collision of a ball (the photon) with a moving wall (the phonon) for the cases of the wall moving towards (in the anti-Stokes process) or away (in the Stokes process) from the ball as they collide. This simple analogy helps illustrate how momentum and energy are exchanged in optomechanical heating (Stokes) and cooling (anti-Stokes) processes.

While these processes naturally lead to a change in phonon population and thereby mode temperature, specific challenges arise in practical demonstration and detection of the phenomena. The most significant challenge is that since we are seeking to address the natural thermal phonons of the medium, we are restricted to a spontaneous Brillouin scattering regime as opposed to stimulated. Stimulated Brillouin scattering techniques are often employed to dramatically increase scattered power and aid in detection (see Appendix \ref{appendix:comparison} and specifically Figure \ref{fig:SponBSvsStimBSvsCoBS} for a comparison of scattered power produced by different Brillouin techniques). However, when stimulated conditions are met, the thermal phonons of the medium are actively driven, often by the injection of an additional optical field. Therefore, the stimulated scattering process is no longer a result of spontaneous thermal phonons but rather of optically driven phonon populations. As explained in Appendix \ref{appendix:comparison}, the condition for achieving stimulated Brillouin scattering is an overall process gain factor, \(G = G_{B}P_{P}L\), much greater than unity \((G \gg 1)\), where \(G_{B}\) is the effective Brillouin gain, \(P_{P}\) is the pump power, and \(L\) is the effective length. An ideal testbed for demonstrating optomechanical cooling would therefore feature an overall process gain factor near but not exceeding unity in order to maximize feasibility of measurement while ensuring a spontaneous scattering regime.

In addition to this fundamental requirement, rate conditions provide additional criticl constraints. The rate at which the phonons are depleted through the optomechanical cooling process must exceed the replenishment rate by the surrounding thermal bath.

Ideal testbed for demonstrating optomechanical cooling:

\begin{enumerate}
\item Large acousto-optic coupling
\item Tight confinement of light and sound
\item Sound speed not too fast (too large brillouin frequency shift, expensive fast elecronics for detection) or too slow (too small brillouin frequency shift, making it hard to separate sideband from carrier)
\item Large single pass gain GPL
\item Fast escape condition - phonon mode depletion rate must exceed repopulation rate \(4v_{g}/L \gg \Gamma\)
\end{enumerate}

%--------------------------------------------------------------------%

\section{Cooling Platform: \texorpdfstring{$CS_{2}$}{CS2}-\acl{LCOF}}
\label{sec:Cooling:Platform}

Demonstration of optomechanical cooling and heating of traveling wave phonons requires a device or material host that [can balance all of the requirements I just layed out in the previous section!] features large optomechanical coupling as well as tight optical and acoustic confinement for large acousto-optic overlap. A \ce{CS2}-filled \ac{LCOF} similar to that used by Behunin et al. (2019) was used.\cite{behunin2019spontaneous}

%--------------------------------------------------------------------%

\section{Experimental Setup}
\label{sec:Cooling:Setup}


\subsection{Main Experiment}
\label{subsec:Cooling:Setup:Main}


\subsection{Pump-Probe Experiment}
\label{subsec:Cooling:Setup:Pump-Probe}

%--------------------------------------------------------------------%

\section{Results}
\label{sec:Cooling:Results}


\subsection{Main Experiment Results}
\label{subsec:Cooling:Results:Main}

need for normalized cooling metric - per pump power, and such that one cannot start a system in a heated state to achieve a greater stated cooling ability. Or maybe room temperature is identical across systems? check. if not, normalize.

\subsection{Pump-Probe Experiment Results}
\label{subsec:Cooling:Results:Pump-Probe}

%--------------------------------------------------------------------%

\section{Discussion}
\label{sec:Cooling:Discussion}

Ideas to achieve net cooling, one might design a system where:

1) single pass gain bias
     swing energy transfer bias to favor anti-Stokes over Stokes (mode-dependent gain)
     the Stokes process were not permitted (ryan's only idea), or just significantly restricted - what is that bias threshold requirement?
     currently, we are net *heating* the system because it is easier to heat from equilibrium than cool (right? explore this. it's also hard to make it hotter beyond a certain point.)
     implementation ideas:
       multi-pump scheme to destructively interfere with Stokes and/or constructively interfere with anti-Stokes
       doping or specialized waveguide gratings that pick out the Stokes band for out-of-plane scattering

2) time bias
     create an energy transfer *rate* bias between Stokes and anti-Stokes
     can be accomplished with either the brillouin energy transfer rate or the repopulation rate

       brillouin process rate:
         4vg/L, have either group velocity or length be different for Stokes vs anti-Stokes (smaller vg or larger L for Stokes than anti-Stokes)
         make anti-Stokes fast light and/or Stokes slow light
         vg inversely proportional to pump power, so it's a balance
           increasing pump power = slower escape time (4vgL), and also larger dissipation (Gamma+)

       repopulation rate:
         thermally insulate the anti-Stokes mode from the thermal bath that would repopulate it
         or at least insulate it some amount more than Stokes (what is that threshold? even if it's only net cooled for picoseconds, what is that minimum crossover point insulation bias point)
         essentially locks in the cooled mode while letting the heated mode spill out
         implementation idea:
           design a fiber/waveguide with acoustic directional bias, such that phonons travelling one direction dissipate quicker because of the geometry/acoustic properties of the fiber. (triangles? acting as tapered acoustic dissipators, pointing in one direction?)

3) starting temperature
     Could you acieve net cooling in a cheap and dirty sense by starting the system in a very heated state, thus invoking a natural dissipation rate bias?! - yes!
     Things do naturally run hot, perhaps no extra fancy engineering is needed for some practical systems? (data processing, reduce thermal noise from above ambient heat)
     not traditional definition of optical refrigeration below ambient/thermal bath, but still *useful*
     is this already done? lit search

Think about a practical device or system for each of these cases (think waveguide playground!)

\subsection{Application to Ground State Cooling}
\label{subsec:Cooling:Discussion:Ground-State}


\subsection{Standardized Cooling Metric}
\label{subsec:Cooling:Discussion:Metric}


\subsection{Tapered chalcogenide Photonic Crystal Fiber: Max Plank Results}
\label{subsec:Cooling:Discussion:Max-Plank}
