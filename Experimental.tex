\chapter{Foundational Experimental Techniques and Instrumentation}
\label{ch:Experimental}
\acresetall

This is an inline citation, \cite{boyd2020nonlinear}. This is a parenthetical citation \citep{boyd2020nonlinear}. This is a figure reference (Figure \ref{fig:cooling-system}). This is a section reference \S\ref{sec:Ch1:Brillouin Scattering}. This is a chapter reference with chapter spelled out: \autoref{chap: 2-Cooling}. This is an acronym definition \ac{AGU}. This is the second time I use the acronym in this section \ac{AGU}. This is if I want to spell out the full acronym again \acf{AGU}. Define new acronyms in the acronyms.tex file.


%%%%%%%%%%%%%%%%%%%%%%%%%%%%%%%%%%%
\section{Experimental Techniques}
\label{sec:Experimental:Experimental Techniques}
\lipsum[1]

  \subsection{ways we can direct light in a photonic system}
  \subsection{photonic devices and diagrams}
  \subsection{ways we can select and isolate signals}
  \subsection{heterodyne detection and the role of the LO}
  \subsection{loss in a photonic system}
  \subsection{free space optics and beam alignment}
  \subsection{special fiber types and properties}

\section{Optical Instrumentation}
\label{sec:Experimental:Optical Instrumentation}
\lipsum[1]

\section{Electronic Instrumentation}
\label{sec:Experimental:Electronic Instrumentation}
\lipsum[1]

\section{Noise and Background Handling}
\label{sec:Experimental:Noise}
\lipsum[1]

\section{Custom Software}
\label{sec:Experimental:Software}
\lipsum[1]

  \subsection{Description of Python Script for CABS Data Collection}
  \label{subsec:Experimental:Software:Python}
  \lipsum[1]

  \subsection{Description of Plotting Data in Go Program}
  \label{subsec:Experimental:Software:Go}
  \lipsum[1]
