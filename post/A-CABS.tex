\doublespacing
\chapter{Supplementary Information for Chapter \ref{ch:CABS}: Manuscript II}
\label{appendix: CABS}
\acresetall

\section{Equal Contribution of Pump, Stokes, and Probe Optical Fields}

Equation \ref{Eq:Theoretical Framework:Scattered Power} gives the somewhat unintuitive result that the powers of the Pump, Stokes, and Probe waves contribute equally to the resulting scattered power of the Signal and invites verification with a miniexperiment. Initially, this experiment was motivated by a practical consideration: determination of whether the placement of a high power amplifier on any specific line of the setup (Pump, Stokes, or Probe) would offer any advantage over another.

To test this, we conducted a controlled experiment with a 1 mm carbon disulfide ($CS_{2}$) sample. For each measurement, one of the three source powers (Pump, Stokes, or Probe) was systematically reduced by 75\% while holding the others constant and ensuring consistent experimental conditions across trials. Table \ref{tab:PSPr-Contribute-Equally} shows the respective powers for each source during the three measurements, along with the multiplicative total contribution of the three powers for each measurement towards the generation of scattered power of the Signal.

\begin{table}[h]
  \centering
  \renewcommand{\arraystretch}{1.2}
  \begin{tabular}{|c|c|c|c|c|}
    \hline
    \textbf{Measurement} & \textbf{Pump Power (mW)} & \textbf{Stokes Power (mW)} & \textbf{Probe Power (mW)} & \textbf{Total (mW$^{3}$)} \\
    \hline
    Pump Lower & 19.190 & 32.210 & 54.560 & 3.372 $\times 10^{4}$ \\
    Stokes Lower & 76.600 & 8.020 & 54.650 & 3.359 $\times 10^{4}$ \\
    Probe Lower & 76.600 & 32.530 & 13.480 & 3.359 $\times 10^{4}$ \\
    \hline
  \end{tabular}
    \caption{Power values for each source (Pump, Stokes, Probe) across the three measurements, with the multiplicative total power for each setup.}
    \label{tab:PSPr-Contribute-Equally}
\end{table}

Figure \ref{fig:PSPr-Contribute-Equally} displays the average results from these three measurements, plotted with error bars representing one standard deviation of the mean. For increased certainty, Figure \ref{fig:PSPr-Contribute-Equally-2sigma} presents the same data with error bars extended to two standard deviations, providing additional confidence in the reproducibility of the results. This experiment confirms that the scattered Signal power indeed depends equally on each of the three contributing wave powers, as expected from the theoretical framework. Consequently, boosting the power of any of the three sources affects the Signal power equally, allowing flexibility in pragmatic design across any of the three lines. Ultimately, this result reinforces the reliability of Equation \ref{Eq:Theoretical Framework:Scattered Power} for predicting Signal power across a range of power distributions within practical settings.

% \begin{figure}[h]
%   \centering
%   \includegraphics[width=.45\textwidth]{PSPr-Contribute-Equally.pdf}
%   \caption{Signal power contributions with error bars representing one standard deviation of the mean for each measurement.}
%   \label{fig:PSPr-Contribute-Equally}
% \end{figure}
%
% \begin{figure}[h]
%   \centering
%   \includegraphics[width=.45\textwidth]{PSPr-Contribute-Equally-2sigma.pdf}
%   \caption{Signal power contributions with error bars extended to two standard deviations of the mean for each measurement.}
%   \label{fig:PSPr-Contribute-Equally-2sigma}
% \end{figure}

\begin{figure}[h]
  \centering
  \begin{subfigure}{0.45\textwidth}
    \centering
    \includegraphics[width=\textwidth]{figs/4-CABS/PSPr-Contribute-Equally.pdf}
    \caption{Signal power contributions with error bars representing one standard deviation of the mean for each measurement.}
    \label{fig:PSPr-Contribute-Equally}
  \end{subfigure}%
  \hfill
  \begin{subfigure}{0.45\textwidth}
    \centering
    \includegraphics[width=\textwidth]{figs/4-CABS/PSPr-Contribute-Equally-2sigma.pdf}
    \caption{Signal power contributions with error bars extended to two standard deviations of the mean for each measurement.}
    \label{fig:PSPr-Contribute-Equally-2sigma}
  \end{subfigure}
  \caption{Comparison of Signal power contributions with error bars representing one (a) and two (b) standard deviations of the mean for each measurement.}
  \label{fig:combined}
\end{figure}


\section{Scattered Power Comparison to Traditional Brillouin Scattering Processes}

This appendix provides a comparative analysis of the scattered power produced by our instrument to that of standard Brillouin scattering processes-that is, spontaneous and stimulated Brillouin scattering. The difference in behavior of our instrumdent from the traditional techniques arises due to the coherent stimulation of the acoustic mode by the pump and Stokes fields, producing a 4-wave-coupled-amplitude interaction that yields much higher scattered powers in smaller lengths. While our instrument is particularly well suited to small interaction lengths due to enhanced phase-matching relaxation, it maintains production of a significant amount of scatterd power at greater lengths as well (greater than 1 meter). This is because the reduction in scattered power from the breakdown of phase-matching relaxation at greater lengths is perfectly counter-balanced by the quadradic dependence on length in the overall scatterd power, as seen in Eq. \ref{Eq:Theoretical Framework:Scattered Power}. At very large lengths (greater than 1 km), the instrument is ultimately limited by the coherence length of the lasers employed, as the process relies on the coherent stimulation of the phonon mode and thus the mutual coherence of the pump and Stokes fields over the interaction length. Here we offer an exploration into the respective performance of each technique across the entire meaningful length scale, from nanometers to kilometers.

Despite shared dependence on basic Brillouin scattering principles, the three techniques compared here (spontaneous-, stimulated-, and coherently stimulated Brillouin scattering) yield significantly different scattered power for identical experimental paramters. At small lengths, the high-gain threshold for optical stimulation of the material fluctuations is often not achievable without the use of extremely large optical powers. This prevents the system from entering a process of exponential growth of the scattered Stokes light indicative of stimulated Brillouin scattering.\cite{boyd2020nonlinear} In this low-gain regime, any scattered Stokes light is spontanously scattered from thermal fluctuations of the material, or from quantum-mechanical fluctuations of materials at the ground state. The low-gain regime is defined by an overall process gain factor, denoted by $G = G_{P}P_{P}L$, which is much less than unity ($G \ll 1$). Here, $G_{B}$ is the effective Brillouin gain in $W^{-1}m^{-1}$ ($G_{B} = \frac{g}{A_{eff}}$), $P_{P}$ is the pump power, and $L$ is the effective length. This spontaneous scattering process follows a linear growth trend described by Boyd et al in 1990\cite{boyd1990noise} as

\begin{equation}
  R = \frac{\langle|E_{S}|^{2}\rangle}{\langle|E_{P}|^{2}\rangle} = (\bar{n} + 1)g\hbar\omega_{S}\Gamma_{B}\frac{L}{4A_{eff}},
\end{equation}

where R is the reflectivity, or the ratio of scattered Stokes intensity to incident pump intensity, and $\bar{n} = (e^{\frac{\hbar\Omega_{B}}{k_{b}T}} - 1)^{-1}$ is the mean number of phonons occupying the mode due to thermal fluctuations of the material. Rearranging this equation and converting to effective Brillouin gain, $G_{B}$, and power by applying the effective area, we arrive at the scattered power of the Stokes spontaneous Brillouin scattering process,

\begin{equation}
  P_S = \frac{1}{4}G_{B}P_{P}L\hbar\omega_{S}\Gamma_{B}(\bar{n} + 1).
  \label{eq:SponBSnbar}
\end{equation}

At room temperature and typical Brillouin frequencies in the GHz range, the quantity $k_{b}T \gg \hbar\Omega_{B}$, allowing

\begin{equation}
e^{\frac{\hbar\Omega_{B}}{k_{b}T}} \approx 1 + \frac{\hbar\Omega_{B}}{k_{b}T}
\end{equation}

to be a good approximation. We thus find that

\begin{equation}
(\bar{n} + 1) \approx \bar{n} \approx \frac{k_{b}T}{\hbar\Omega_{B}}.
\end{equation}

Inserting this reduced quantity into Eq. \ref{eq:SponBSnbar}, we arrive at a convenient expression for the scattered power of the Stokes spontaneous Brillouin scattering process,

\begin{equation}
  P_{S, \,\textit{SponBS}} = \frac{G_{B}P_{P}L\omega_{S}\Gamma_{B}k_{b}T}{4\Omega_{B}}.
\end{equation}

It may be noted that the derived expression for the low-gain spontaneous regime here matches the form reported by Kharel et al. in 2016 \cite{kharel2016noise} for the complementary forward scattering processe. While the two scenarios—our backward scattering geometry versus the forward scattering geometry discussed by Kharel et al.—differ in directionality, the underlying physics of light coupling to thermally excited acoustic modes is the same and reflects the fundamental similarity in how thermal phonons mediate the interaction between optical fields in the low-gain (spontaneous) regime.

Next we turn to the high-gain regime leading to a stimulated Brillouin scattering process. This regime is defined by an overall process gain factor, $G = G_{P}P_{P}L$, that is much greater than unity ($G \gg 1$). For organic liquids, this crossover threshold from spontaneous to stimulated regimes occurs in the range of $20 < G < 25,$\cite{boyd1990noise} whereas for typical lengths of single mode fiber it can be lower\cite{ippen1972stimulated} owing to the small effective area compared to longer effective lengths of fiber typically used.

The reflectivity of a stimulated Brillouin scattering process in the high-gain regime is given by \cite{boyd1990noise}

\begin{equation}
  R = \frac{\langle|E_{S}|^{2}\rangle}{\langle|E_{P}|^{2}\rangle} = \frac{Y}{\sqrt{\pi}}\frac{e^{G}}{G^{\frac{3}{2}}},
\end{equation}

where $Y$ is the reflectivity of the low-gain (spontaneous) regime given above and $G$ is the overall process gain factor, $G = G_{P}P_{P}L$. Again, converting to the effective Brillouin gain, $G_{B}$, and power by applying the effective area, we solve for the scattered power of the Stokes field,

\begin{equation}
  P_{S, \,\textit{StimBS}} = \frac{G_{B}P_{P}L\omega_{S}\Gamma_{B}k_{b}T}{4\sqrt{\pi}\Omega_{B}}\frac{e^{G}}{G^{\frac{3}{2}}}
  \label{eq:StimBSUndepletedPump}
\end{equation}

This expression captures the exponential growth in scattered power as any parameter within the overall process gain factor, $G = G_{B}P_{P}L$, increases. However, this exponential growth can only continue while the pump is not significantly undepleted. Once the scattered power described by Eq. \ref{eq:StimBSUndepletedPump} grows to a significant fraction of the driving pump power, the exponential increase in scattered Stokes power asymptotically approaches the pump power. For very large $G$, virtually all of the pump energy is converted to scattered Stokes energy in a complete transfer process.\cite{boyd2020nonlinear} To account for pump depletion, we numerically solve the transendental equation derived in Boyd's Nonlinear Optics which describes the effects of pump depletion, given here in terms of power as

\begin{equation}
  P_S(L) = \frac{P_S(0)x(1 - x)}{e^{G_{B}P_{P}(0)L(1 - x)} - x},
\end{equation}

where $x = P_S(0)/P_P(0)$, or the ratio of the unknown Stokes power at the end of its journey through the medium ($z=0$) to the known pump power at the beginning (also $z=0$). This solution for $x$, specific to system parameters such as length, offers via its definition the solution to the unknown power of the scattered Stokes light at the end of its traversal through the effective length, given as

\begin{equation}
  P_S(0) = xP_{P}(0).
\end{equation}

The solution to this numeric approach to scattered power in the high-gain (stimulated) Brillouin scattering regime with pump depletion effects at large $G$ is plotted for varying effective lengths in Fig. \ref{fig:SponBSvsStimBSvsCoBS}, along with the analytical solutions derived previously for the low-gain (spontaneous) regime and our coherently stimulated Brillouin spectrometer given by Eq. \ref{Eq:Theoretical Framework:Scattered Power}. System parameters used to generate the plot for each of the three processes are provided in Tables \ref{tab:CoBS Parameters} and \ref{tab:SBS Parameters}. Wherever possible, the parameters shared by all three Brillouin scattering processes were kept consistent, while quantities unique to each process were assigned their respective values.

\begin{table}[h]
  \centering
  \textbf{Coherently Stimulated Brillouin Scattering Process Model System Parameters}
  \renewcommand{\arraystretch}{1.2}
  \begin{tabular}{|c|c|c|c|c|}
    \hline
    $G_{B}$ & $P_{P}$ & $P_{S}$ & $P_{Pr}$ & $\Delta\lambda$ \\
    \hline
    0.6 $W^{-1} m^{-1}$ & 1 $W$ & 1 $W$ & 1 $W$ & 20 pm \\
    \hline
  \end{tabular}
  \caption{Parameters relevant to the coherently stimulated backward Brillouin scattering process for the example UHNA3 fiber. $G_{B}$ is the effective Brillouin gain, $P_{P}$ is the pump power, $P_{S}$ is the Stokes power, $P_{Pr}$ is the probe power, and $\Delta\lambda$ is the wavelength detuning of the probe from the pump.}
  \label{tab:CoBS Parameters}
\end{table}

\begin{table}[h]
  \centering
  \textbf{Spontaneous and Stimulated Scattering Process Model System Parameters}
  \renewcommand{\arraystretch}{1.2}
  \begin{tabular}{|c|c|c|c|c|c|c|c|c|}
    \hline
    $G_{B}$ & $P_{P}$ & $P_{S,seed}$ & $n$ & $\lambda_P$ & $\Gamma_{B}$ & $k_{B}$ & $T$ & $\Omega_{B}$ \\
    \hline
    0.6 $W^{-1} m^{-1}$ & 1 $W$ & $1$ pW & 1.48 & $1549$ nm & $2\pi \cdot 80$ MHz & $1.38 \times 10^{-23}$ J/K & 295 K & $2\pi \cdot 9.18$ GHz \\
    \hline
  \end{tabular}
  \caption{Parameters relevant to the spontaneous and/or stimulated backward Brillouin scattering processes for the example UHNA3 fiber. $G_{B}$ is the Brillouin gain coefficient, $P_{P}$ is the pump power, $\omega$ is the optical angular frequency, $\Gamma_{B}$ is the acoustic damping rate, $k_{B}$ is Boltzmann's constant, $T$ is the temperature, and $\Omega_{B}$ is the acoustic angular frequency.}
  \label{tab:SBS Parameters}
\end{table}

At lengths beyond a centimeter, the phase-matching relaxation of the coherently stimulated process begins to break down, and the specific choice in pump and probe detuning becomes critical. This corresponds to a narrowing of the $sinc^{2}$ function given in Eq. \ref{Eq:Phi}. The scattered power beyond this length rises and falls according to the oscillations of the $sinc^{2}$ function far from the origin. As length increases continuously beyond 1 meter, the scattered power oscillates with increasing frequency and ceases to offer practical significance. To better visualize the scattered power offered by the instrument in this region, we have computed the envelope of scattered power. In a laboratory setting, the appropriate pump and probe detuning would be selected for the specific sample length being measured such that the scattered power function lies on a local peak of the $sinc^{2}$ function.

\begin{figure}[t]
\centering
\includegraphics[width=\textwidth]{figs/4-CABS/SponBSvsStimBSvsCoBS.pdf}
\caption{Comparison of scattered power from a spontaneous Brillouin scattering process and our coherently stimulated Brillouin spectrometer.}
\label{fig:SponBSvsStimBSvsCoBS}
\end{figure}

Fig. \ref{fig:SponBSvsStimBSvsCoBS} shows the advantage that our coherently stimulated Brillouin spectrometer offers compared to the traditional Brillouin processes for the example medium of UHNA3 fiber. For lengths up to about 50 meters and down to as low as 100 nanometers, the coherently stimulated process employed by our instrument offers superior scattered power, with the relative advantage peaking for a length just under 1 cm. At this length, the gain factor $G$ places the traditional process within the low-gain (spontaneous) regime, and thus the scattered power generated is only on the order of 10s of picoWatts. In contrast, the scattered power for the same system offered by our instrument is on the order of 10s of microWatts, exceeding that of the spontenous process by a factor of a million. This is, of course, the most ideal case for this system, however it can be seen from Fig. \ref{fig:SponBSvsStimBSvsCoBS} that the coherently stimulated process offers orders of magnitude more scattered power than either traditional process through a wide range of lengths.

\section{Data}
